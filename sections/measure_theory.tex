\chapter{Measure Theory}
\section{Measure in Euclidean Space}
\begin{definition}
    A rectangle $C^a_b$ in $\bb{R}^n$ is a closed set given by 
    \(
        C^a_b = [a_1,b_1]\times \dots \times [a_n\times b_n]
    \). The volume of $C^a_b$, denoted by $\vol(C^a_b)$, is defined to be 
    \(
      \displaystyle  \vol(C^a_b) = \prod_{i = 1}^n (b_i - a_i)
    \). A union of rectangles is said to be \textit{almost disjoint} if the interior of rectangles are disjoint.
\end{definition}

\begin{lemma}
    If $C^a_b = \displaystyle\bigsqcup_{k = 1}^nR_k$, then 
    \(
      \displaystyle  C^a_b= \sum_{k = 1}^n \vol(R_k)
    \).
\end{lemma}

\begin{theorem}
    Every open subset $U$ of $\bb{R}$ can be writen uniquely as a countable union of disjoint open intervals.
\end{theorem}
\begin{definition}[Exterior Measure]    
    Let $A \subseteq \bb{R}^n$, the \textit{outer measure} of $A$ is 
    \[
        m^*(A) = \inf\left\{\sum_{i}\vol(C_i)\mid \{C_i\} \text{ covers } A\right\}
    \]
    
\end{definition}
\begin{proposition}
     If $A \subseteq B$, then $m^*(A) \leq m^*(B)$.
\end{proposition}
\begin{proof}
     Since every $\{C_i\}$ covering $B$ also covers $A$, then $m^*(A) \leq m^*(B)$ follows from the infimum property.
\end{proof}

\begin{proposition}
    Let $A,B \subseteq \bb{R}^n$, then $m^*(A\cup B)  \leq m^*(A) + m^*(B)$, the equality holds if and only if $d(A,B) > 0$.
\end{proposition}
\begin{proof}
    Without loss of generality, assume that $A$ and $B$ are bounded. Given $\ep > 0$, then there exists finite coverings $\{C_i\}$ of $A$ and $\{D_j\}$ of $B$ by cubes such that 
    \[
        m^*(A) + \ep \geq \sum_{i}\vol(C_i) \text{ and }m^*(B) + \ep \geq \sum_{j}\vol(D_j).
    \]Then every element $x \in A \cup B$ must be contained in some $C_i \cup D_j$, then $\{C_i\} \cup \{D_j\}$ is a covering of $A \cup B$ by cubes.
 it follows that 
    \begin{equation}
        \begin{aligned}
             m^*(A \cup B ) & \leq \sum_i \vol(C_i) + \sum_j\vol(D_j) \leq  m^*(A) + m^*(B) + 2\ep
        \end{aligned}
    \end{equation}
    Since $\ep$ was arbitrary, the following inequality holds. Now we suppose $d(A,B) = \delta >0$,  for every $\ep < \delta$, there exists a covering of $A \cup B$ by cubes of size $\ep$ such that it can be decomposed into two finite coverings $\{C_i\}$ of $A$ and $\{D_i\}$ of $B$, and we have
    \[
        m^*(A\cup B) + \ep \geq \sum_i\vol(C_i) + \sum_j \vol(D_j)
    \] Then it follows that 
    \begin{equation}
        m^*(A) + m^*(B) \leq \sum_{i} \vol(C_i) + \sum_{j} \vol(D_j) \leq m^*(A \cup B ) + \ep
    \end{equation}
        

    As every $\ep$ is arbitrarily small, it follows from $(6.2)$ that
    \[
    m^*(A) + m^*(B) \leq m^*(A \cup B),
    \]
    Hence, the equality holds, as desired.
\end{proof}
\begin{proposition}
    If $E = \bigcup_{j = 1}^\infty E_j$, then $m^*(E) \leq \dsum_j m^*(E_j)$
\end{proposition}
\begin{proof}
    If there is some $j \in \bb{N}$ such that $m^*(E_i) = \infty$, then the inequality is trivial. Now we suppose that each $m^*(E_i) < \infty$.  For each $j$, there exists a covering $\{C_{k}^j\}$ of $E_j$ by cubes such that 
\[
    \sum_{k}\vol(C_{k}^j) \leq m^*(E_j) + \frac{\ep}{2^j}
\]
Since $E = \bigcup_j E_j$ and $E_j \subseteq \bigcup_k C^j_k$, then we have $E \subseteq \bigcup_{j,k}C^j_k$, it follows that 
\begin{equation}
    \begin{aligned}
        m^*(E) &\leq \sum_{k,j}\vol(C_k^j) = \sum_{j = 1}^{\infty} \sum_k \vol(C_k^j)\\
        &\leq \sum_{j = 1}^{\infty} \left(m^*(E_j) + \frac{ \ep}{2^j}\right)\\
        &= \left( \sum_{j = 1}^{\infty}  m^*(E_j)\right) + \ep.
    \end{aligned}
\end{equation}
    

Since $\ep > 0$ was arbitrary, then it yields the claim follows from $(6.3)$, as desired
\end{proof}
\begin{proposition}
    If $E \subseteq \bb{R}^n$, then 
    \[
        m^*(E) = \inf\{m^*(U)\mid U \supseteq E \text{ open}\}
    \]
\end{proposition}
\begin{proof}
    If $V\supseteq E$ is an open set, by monotonicity, we have $m^*(E) \leq m^*(V)$. To prove the converse inequality, let $\ep > 0$ be arbitrary, choose a covering $\{C_i\}$ of $E$ by cubes such that 
    \[
        \sum_i \vol(C_i) \leq m^*(E) + \frac{\ep}{2}
    \]
    For each $i$, let $C_i^0$ be an open cube containing $C_i$ such that \[\vol(C_i^0) \leq \vol(C_i) + \frac{\ep}{2^{i + 1}}.\] Then $U = \bigcup_i C^0_i$ is an open set containing $E$, then we have 
    \begin{align*}
        m^*(U) &\leq \sum_i \vol(C_i^0)\leq \sum_i \left(\vol(C_i) + \frac{\ep}{2^i}\right) = \left(\sum_i\vol(C_i)\right) + \frac{\ep}{2} \leq  m^*(E) + \ep
    \end{align*}
    Moreover, since we have 
    \[
        m^*(U) \geq \inf_{V \supseteq E \text{ open}}m^*(V),
    \] then we obtain 
    \[
        m^*(E) +\ep \geq \inf_{V \supseteq E\text{ open}} m^*(V)
    \]
    Since $\ep > 0$ was abitrary, combining with the reverse inequality yields the equality.
\end{proof}
\begin{proposition}
    If $E = \bigsqcup_{j = 1}^\infty C_i$, then 
    \[
        m^*(E) = \sum_{j = 1}^\infty\vol(C_i)
    \]
\end{proposition}
\section{Lebesgue measure}
\begin{definition}[Lebesgue measurable sets]
    A subset $E \subseteq \bb{R}^n$ is said to be \textit{Lebesgue measurable} if for any $\ep > 0$, there exists an open set $U \supseteq E$ such that 
    \[
            m^*(U \backslash E) < \ep
    \]
    If $E$ is measurable, then we define its \textit{Lebesgue measure} $m(E)$ by 
    \[
        m(E) = m^*(E)
    \]
\end{definition}
\begin{definition}
    A subset $X \subseteq \bb{R}^n$ is said to have \textit{measure zero} if for every $\ep > 0$, there exists a countable cover of $X$ by open rectangles $\{C_i\}$ such that 
    \[
         \sum_i \vol(C_i) < \ep.
    \]
\end{definition}
\begin{proposition}
    Every open set $\bb{R}^n$ is countable.
\end{proposition}
\begin{proposition}
    If $m^*(E) = 0$, then $E$ is measurable. In particular, if $F$ is a subset of a set of exterior measure $0$, then $F$ is measurable.
\end{proposition}
\begin{proof}
    Let $\ep > 0$ be arbitrary, then there exists a covering $\{C_i\}$ of $E$ such that 
    \[
    \sum_i \vol(C_i) \leq \frac{\ep}{2}
    \]
    For each $i$, let $C^0_i$ be an open cube containing $C_i$ such that $\vol(C^0_i) \leq \vol(C_i) + \frac{\ep}{2^{i + 1}}$. Then $U = \bigcup_i C^0_i$ is an open set containing $E$. By monotonicity, we have 
    \begin{align*}
        m^*(U \backslash E) &\leq m^*(U)\leq \sum_i \vol(C_i^0)\leq \sum_i\left(\vol(C_i) + \frac{\ep}{2^{i + 1}}\right)\\
        &= \sum_i\vol(C_i) + \frac{\ep}{2} \leq \frac{\ep}{2} + \frac{\ep}{2} = \ep
    \end{align*}
     Thus $E$ is measurable. For any $F \subseteq E$, monotonicity implies $m^*(F) \leq m^*(E) = 0$. Applying the same argument as above, $F$ is  measurable.
\end{proof}
\begin{proposition}
    A countable union of measurable sets is measurable.
\end{proposition}
\begin{proposition}
    Closed sets are measurable.
\end{proposition}
\begin{lemma}
    If $F$ is closed, $K$ is compact, and these sets are disjoint, then \[d(F,K) > 0\]
\end{lemma}
\begin{proof}
    Since $F$ is closed, for every $x \in K$, there exists $\delta_x > 0$ such that 
    \[
    d(x,F) > 3\delta_x
    \]
    Since $X = \bigcup_{x \in K}B(x,2\delta_x)$ is an open cover of $K$ and $K$ is compact, then $X$ attains a finite subcover $X_N = \bigcup_{n = 1}^NB(x_n,2\delta_{x_n})$. Let $\delta = \min\{\delta_{x_1},\dots,\delta_{x_N}\}$, it follows that $d(F,K) \geq d(F,X_N) \geq \delta$. Moreover, let $x \in F$ and $y \in K$, then there exists $j < N$ such that $d(x,x_j) \leq 2\delta_j$, we have 
    \[
        d(x,y) \geq  d(x_n,y) - d(x,x_n) \geq 3\delta_j - 2\delta_j = \delta_j \geq \delta
    \]
    Hence, we are done.
\end{proof}
\begin{proposition}
    The complement of a measurable set is measurable.
\end{proposition}
\begin{proposition}
    A countable intersection of measure sets is measurable.
\end{proposition}
\begin{proposition}
    If $E_1,E_2,\dots$ are disjoint measurable sets, and $E = \bigcup_j E_j$, then 
    \[
        m(E) = \sum_j m(E_j)
    \]
\end{proposition}
\begin{proposition}
    Let $(E_n)$ be measurable subset of $\bb{R}^n$.
    \begin{enumerate}
        \item If $(E_n)$ is increasing to $E$, then $m(E) = \nlim m(E_n)$.
        \item If $(E_n)$ is decreasing to $E$ and $m(E_n) < \infty$ for some $n$, then $m(E) = \nlim m(E_n)$.
    \end{enumerate}
\end{proposition}
\begin{proof}
    1. Let $(G_n)$ be a sequence of subsets defined by $G_1 = E_1$ and $G_n = E_{n} \backslash E_{n -1}$ for all $n \geq 2$. Observe that $(G_n)$ is pairwise disjoint, measurable and we have 
$E = \bigcup_j G_j$. Then $E$ is measurable whose measure is calculated by 
\begin{align*}
     m(E) &= \sum_{j = 1}^{\infty}m(G_j) =\nlim  \sum_{j = 1}^{n}m(G_j) \\
     &= \nlim \left(\sum_{j = 2}^n m(E_n) - m(E_{n - 1})\right) + m(E_1)\\
    &=\nlim m(E_n)
\end{align*}
   
2. Assume that $m(E_1) < \infty$. Let $(G_n)$ be a sequence defined by $G_1 = E_1$ and $G_n = E_{n - 1} \backslash E_{n}$ for all $n \geq 2$, then $(G_n)$ is also pairwise disjoint, measurable and we have 
\[
    E_1 = E \cup \bigcup_j G_j
\]
Since $E \cap G_j=\varnothing$  for all $j$, it follows that 
\begin{align*}
    m(E_1) &= m(E) + \nlim \sum_{j = 1}^n m(G_j)\\
    &= m(E) + \nlim \left(\sum_{j = 2}^n m(E_{j - 1}) - m(E_{j})\right) + m(E_1)\\
    &= m(E) + m(E_1) - \nlim m(E_{n})
\end{align*}
Thus $m(E) = \nlim m(E_{n})$, as desired.
\end{proof}
\section{Measurable functions}
\begin{definition}
    A function $f: E \to \bb{R}^n$ is said to be \textit{measurable} if for all $a \in \bb{R}$, the set 
    \[
        \{ f < a\} := f^{-1}([-\infty,a)) 
    \]
    is measurable.
\end{definition}
\begin{proposition}
    The function $f: E \to \bb{R}^n$ is measurable if and only if one of the followings hold:
    \begin{enumerate}
        \item $\{f < a\}$ is measurable.
        \item $\{f \leq a\}$ is measurable.
        \item $\{f > a\}$ is measurable.
        \item $\{f \geq a\}$ is measurable.
    \end{enumerate}
\end{proposition}
\begin{proposition}
    If $f$ is continuous on $\bb{R}^n$, then $f$ is measurable. If $f$ is measurable and finite-valued, and $\Phi$ is continuous, then $\Phi \circ f$ is measurable.
\end{proposition}
\begin{proposition}
    If $(f_n)$ is a sequence of measurable functions, then 
    \[
        \sup f_n(x), \quad \inf f_n(x), \quad\lim_{n \to +\infty} \sup f_n(x) \quad \text{and} \quad \nlim \inf f_n(x)
    \]
    are measurable.
\end{proposition}
\begin{proposition}
    If $(f_n)$ is a sequence of measurable functions and $f_n \to f$, then $f$ is measurable.
\end{proposition}
\begin{proposition}
    If $f$ and $g$ are measurable, then 
    \begin{enumerate}
        \item $f^k$ is measurable for $k \in \bb{N}$.
        \item $f + g$ and $fg$ are measurable if both $f$ and $g$ are finite-valued.
    \end{enumerate}
\end{proposition}
\section{Problems}
\begin{problem}
    Prove that the Cantor set $C$ constructed in the text is totally disconnected and perfect. In other words, given two distinct points $x, y \in C$, there is a point $z \notin C$ that lies between $x$ and $y$, and yet $C$ has no isolated points.
\end{problem}
\begin{problem} The Cantor set $C$ can also be described in terms of ternary expansions.
    \begin{enumerate}
        \item Every number in [0, 1] has a ternary expansion
$$
x = \sum_{k=1}^{\infty} a_k 3^{-k}, \text{ where } a_k = 0, 1, \text{ or } 2.
$$
Note that this decomposition is not unique since, for example, $1/3 = \sum_{k=2}^{\infty} 2/3^k$.
Prove that $x \in C$ if and only if $x$ has a representation as above where every $a_k$ is either 0 or 2.
        \item The Cantor-Lebesgue function is defined on $C$ by
        $$
        F(x) = \sum_{k=1}^{\infty} \frac{b_k}{2^k} \text{ if } x = \sum_{k=1}^{\infty} a_k 3^{-k}, \text{ where } b_k = a_k/2.
        $$
        In this definition, we choose the expansion of $x$ in which $a_k = 0$ or 2.
        Show that $F$ is well defined and continuous on $C$, and moreover $F(0) = 0$ as well as $F(1) = 1$.

        \item Prove that $F: C \rightarrow [0, 1]$ is surjective, that is, for every $y \in [0, 1]$ there exists $x \in C$ such that $F(x) = y$.

        \item One can also extend $F$ to be a continuous function on [0, 1] as follows. Note that if $(a, b)$ is an open interval of the complement of $C$, then $F(a) = F(b)$.
        Hence we may define $F$ to have the constant value $F(a)$ in that interval.
    \end{enumerate}

\end{problem}
\begin{problem}
    If $\delta = (\delta_1, ..., \delta_d)$ is a $d$-tuple of positive numbers $\delta_i > 0$, and $E$ is a subset of $\mathbb{R}^d$, we define $\delta E$ by
$$
\delta E = \{(\delta_1 x_1, ..., \delta_d x_d) : \text{where } (x_1, ..., x_d) \in E\}.
$$
Prove that $\delta E$ is measurable whenever $E$ is measurable, and
$$
m(\delta E) = \delta_1 \cdots \delta_d m(E).
$$
\end{problem}
\begin{problem}
    Suppose $L$ is a linear transformation of $\mathbb{R}^d$. Show that if $E$ is a measurable subset of $\mathbb{R}^d$, then so is $L(E)$, by proceeding as follows:

\begin{enumerate}
    \item  Note that if $E$ is compact, so is $L(E)$. Hence if $E$ is an $F_\sigma$ set, so is $L(E)$.
    \item Because $L$ automatically satisfies the inequality
$$
|L(x) - L(x')| \leq M|x - x'|
$$
for some $M$, we can see that $L$ maps any cube of side length $\ell$ into a cube of side length $c_d M \ell$, with $c_d = 2\sqrt{d}$. Now if $m(E) = 0$, there is a collection of cubes $\{Q_j\}$ such that $E \subset \bigcup_j Q_j$, and $\sum_j m(Q_j) < \epsilon$. Thus $m_*(L(E)) \leq c' \epsilon$, and hence $m(L(E)) = 0$. Finally, use Corollary 3.5.
\end{enumerate}
\end{problem}
\begin{problem}
    The purpose of this exercise is to show that covering by a *finite* number of intervals will not suffice in the definition of the outer measure $m_*$.
The outer Jordan content $J_*(E)$ of a set $E$ in $\mathbb{R}$ is defined by
$$
J_*(E) = \inf \sum_{j=1}^{N} |I_j|,
$$
where the inf is taken over every *finite* covering $E \subset \bigcup_{j=1}^{N} I_j$, by intervals $I_j$.

\begin{enumerate}
 \item Prove that $J_*(E) = J_*(\overline{E})$ for every set $E$ (here $\overline{E}$ denotes the closure of $E$).
 \item Exhibit a countable subset $E \subset [0,1]$ such that $J_*(E) = 1$ while $m_*(E) = 0$.
\end{enumerate}
\end{problem}
\begin{problem}[The Borel-Cantelli lemma]
    Suppose $\{E_k\}_{k=1}^{\infty}$ is a countable family of measurable subsets of $\mathbb{R}^d$ and that
$$
\sum_{k=1}^{\infty} m(E_k) < \infty.
$$
Let
$$
\begin{aligned}
E &= \{x \in \mathbb{R}^d : x \in E_k, \text{ for infinitely many } k \} \\
&= \limsup_{k \to \infty} (E_k).
\end{aligned}
$$
\begin{enumerate}
    \item Show that $E$ is measurable.
    \item Prove $m(E) = 0$.
\end{enumerate}
[Hint: Write $E = \bigcap_{n=1}^{\infty} \bigcup_{k \geq n} E_k$.]
\end{problem}
\begin{problem}
    Let $\{f_n\}$ be a sequence of measurable functions on $[0, 1]$ with $|f_n(x)| < \infty$ for a.e $x$. Show that there exists a sequence $c_n$ of positive real numbers such that
$$
\frac{f_n(x)}{c_n} \to 0 \quad \text{a.e. } x
$$

[Hint: Pick $c_n$ such that $m(\{x : |f_n(x)/c_n| > 1/n\}) < 2^{-n}$, and apply the Borel-Cantelli lemma.]
\end{problem}
\begin{problem}
    Given an irrational $x$, one can show (using the pigeon-hole principle, for example) that there exists infinitely many fractions $p/q$, with relatively prime integers $p$ and $q$ such that
$$
\left|x - \frac{p}{q}\right| \leq \frac{1}{q^2}.
$$
However, prove that the set of those $x \in \mathbb{R}$ such that there exist infinitely many fractions $p/q$, with relatively prime integers $p$ and $q$ such that
$$
\left|x - \frac{p}{q}\right| \leq \frac{1}{q^3} \quad (\text{or } \leq 1/q^{2+\epsilon}),
$$
is a set of measure zero.
\end{problem}
\begin{problem}
    Complete the following outline to prove that a bounded function on an interval $[a, b]$ is Riemann integrable if and only if its set of discontinuities has measure zero. This argument is given in detail in the appendix to Book I.

Let $f$ be a bounded function on a compact interval $J$, and let $I(c,r)$ denote the open interval centered at $c$ of radius $r > 0$. Let $osc(f, c, r) = \sup |f(x) - f(y)|$, where the supremum is taken over all $x, y \in J \cap I(c,r)$, and define the oscillation of $f$ at $c$ by $osc(f, c) = \lim_{r \to 0} osc(f, c, r)$. Clearly, $f$ is continuous at $c \in J$ if and only if $osc(f, c) = 0$.

Prove the following assertions:

(a) For every $\epsilon > 0$, the set of points $c$ in $J$ such that $osc(f, c) \geq \epsilon$ is compact.

(b) If the set of discontinuities of $f$ has measure 0, then $f$ is Riemann integrable.

[Hint: Given $\epsilon > 0$ let $A_{\epsilon} = \{c \in J : osc(f, c) \geq \epsilon \}$. Cover $A_{\epsilon}$ by a finite number of open intervals whose total length is $\leq \epsilon$. Select an appropriate partition of $J$ and estimate the difference between the upper and lower sums of $f$ over this partition.]
\end{problem}
\begin{problem}
    Suppose $E$ is measurable with $m(E) < \infty$, and
$$
E = E_1 \cup E_2, \quad E_1 \cap E_2 = \emptyset.
$$
If $m(E) = m_*(E_1) + m_*(E_2)$, then $E_1$ and $E_2$ are measurable.

In particular, if $E \subset Q$, where $Q$ is a finite cube, then $E$ is measurable if and only if $m(Q) = m_*(E) + m_*(Q-E)$.

\end{problem}
\begin{problem}
    Suppose $A$ and $B$ are open sets of finite positive measure. Then we have equality in the Brunn-Minkowski inequality (8) if and only if $A$ and $B$ are convex and similar, that is, there are a $\delta > 0$ and an $h \in \mathbb{R}^d$ such that
$$
A = \delta B + h.
$$
\end{problem}