\section{Embeddings}
\subsection{Immersion and Submersion}
\begin{definition}
    Let $M^m$ and $N^n$ be smooth manifolds, and $F: M \to N$ is a smooth map. For abitrary point $p \in M$, we define \text{the rank of $F$ at $p$} to be the rank of the differential $dF_p$. If $\rank (dF_p) = r$ for all $p \in M$, we say $F$ \textit{has the constant rank $r$}.
\end{definition}
\begin{lemma}
    Let $F: M^m \to N^n$ be a smooth map. For abitrary $p \in M$, $dF_p$ is injective if and only if $\rank(dF_p) = n$ and is surjective if and only if $\rank(dF_p) = m$. Consequently, $dF_p$ is bijective if and only if $\rank(dF_p)= m = n$.
\end{lemma}
\begin{definition}
    Let $F: M^m \to N^n$ be the smooth map. Then $F$ is called a \text{smooth immersion} if its differential is injective at each point, and is called a \text{smooth submersion} if its is surjective at each point. 
\end{definition}
\begin{proposition}
    Injectivity and surjectivity of the differential implies local immersion and submersion, respectively.
\end{proposition}
\begin{theorem}[Inverse Function Theorem on Manifolds]
    Let $F: M^m \to N^n$ be a smooth map. If $p \in M$ is point such that $dF_p$ is invertable, then there are connected neighborhoods $U_p$ and $V_{F(p)}$ such that $F|_{U_p}: U_p \to V_{F(p)}$ is a diffeomorphism.
\end{theorem}
\subsection{Constant Rank Theorem}
\begin{theorem}[Constant Rank Theorem]
    Let $M^m$ and $N^n$ be smooth manifolds and $F: M \to N$ be a smooth map with constant rank $r$. For each $p \in M$, there exists smooth charts $(U,\varphi)$ for $M$ containing $p$ and $(V,\psi)$ for $N$ containing $F(p)$ such that $F(U) \subseteq V$, in which $F$ has a coordinate representation of the form 
    \begin{equation}\label{thm:rank}
        \psi \circ F \circ \varphi^{-1}(x^1,\dots, x^r,\dots,x^m) = (x^1,\dots,x^r,0,\dots,0)
    \end{equation}
\end{theorem}
\begin{proof}
    Since the theorem is local, one can view $M$ and $N$ as open subsets $U$ and $V$ in $\bb{R}^m$ and $\bb{R}^n$, respectively.
    For the sake of condition, we can reorder coordinate such that the $r \times r$ block of $DF_p$ has nonzero determinant. Consider the indentification 
    \begin{align*}
     (x,y) &\lra (x^1,\dots,x^r,y^1,\dots,y^{m-r}),\\
    (u,v) &\lra  (u^1,\dots,u^r,v^1,\dots,v^{n-r}).
    \end{align*}
    Then one can view $F(x,y) = (F_1(x,y),F_2,(x,y))$ for some smooth map $F_1: U \to \bb{R}^r$ and $F_2: U \to \bb{R}^{n - r}$. Let $p=(x_p,y_p)$ and $\varphi: U \to \bb{R}^m$ satisfying $\varphi(x,y) = (F_1(x,y),y)$, we have 
    \[
        D\varphi(x_p,y_p)= \begin{bmatrix}
            DF_1(x,y) & \dfrac{\pa F_1}{\pa y}(x,y)\\
            0 & \delta^i_j
        \end{bmatrix}
    \]
    The above assumption implies that $DF_1(x,y)$ and $\delta_i^j$ are invertable, then $D\varphi(x_p,y_p)$ is invertable.  Apply the inverse function theorem for the map $\varphi$ and $p \in U$, there exists connected neighborhoods $U_p$ such that $\varphi|_{U_p}: U_p \to \varphi(U_p)$ is a diffeomorphism. Then we can consider the local chart $(U_p,\varphi)$ which is a restricted diffeomorphism. On the following restriction, we can let $A,B: \varphi(U_p)\to U_p$ be the smooth function satisfying $\varphi^{-1}(x,y) = (A(x,y),B(x,y))$, then we obtain 
    \[
        (x,y) = \varphi(A(x,y),B(x,y)) = (F_1(A(x,y),B(x,y)),B(x,y)).
    \]  
    It follows that $B(x,y) = y$ and $F_1(A(x,y),y) = x$. Hence $F\circ \varphi^{-1}$ is calculated by 
    \[
        F \circ \varphi^{-1}(x,y) = F(A(x,y),y) = (F_1(A(x,y),y),F_2(A(x,y),y)) = (x, C(x,y)),
    \]
    where $C: \varphi(U_p) \to \bb{R}^{n - r}$ defined by $C(x,y) = F_2(A(x,y),y)$. Moreover, since the Jacobian of $F \circ \varphi^{-1}$ at abitrary point $(x,y) \in \varphi(U_p)$ is 
    \[
        D(F\circ \varphi^{-1})(x,y) = \begin{bmatrix}
            \delta^i_j & 0\\
            \dfrac{\pa C^i}{\pa x^j}(x,y) & \dfrac{\pa C^i}{\pa y^j}(x,y)
        \end{bmatrix},
    \]
    where its rank remains at $r$ and the block $\delta^i_j$ has the rank $r$. Then the block $\dfrac{\pa C^i}{\pa y^j}(x,y)$ must vanish to zero. Therefore $C(x,y)$ is independent of $y$ hence we can set $S(x) = C(x,y) = C(x,y_p)$ for all $y$ and then we obtain 
    \begin{equation}\label{thm:3}
        F \circ \varphi^{-1}(x,y) = (x,S(x))
    \end{equation}
        
    
    Let $V_p \subseteq V$ be an open subset satisfying $V_p = \{(u,v) \in V \mid (u,\varphi^2(p)) \in \varphi(U_p)\}$. Then $V_p$ is a neighborhood of $\varphi(p)$ and the definition of $F \circ \varphi^{-1}$ in ~\ref{thm:3} implies that $F \circ \varphi^{-1}(\varphi(U_p)) = F(U_p) \subseteq V_p$. Then we can consider the local chart $(V_p, \psi)$ such that $\psi: V_p \to \bb{R}^n$ satisfying $\psi(u,v) = (u,v - S(u))$. Notice that $(V_p,\psi)$ is a smooth chart, following from ~\ref{thm:3}, we thus have 
    \[
        \psi \circ F \circ \varphi^{-1}(x,y) = \psi(x,S(x)) = (x, S(x)- S(x)) = (x,0).
    \]
    Since $(\varphi,U_p)$ and $(\psi,V_p)$ satisfy the following conditions, the proof is done.
\end{proof}
\begin{corollary} 
    Let $F: M^m \to N^n$ be a smooth map between smooth manifolds and suppose $M$ is connected. Then the followings are equivalent:
    \begin{enumerate}
        \item For each $p \in M$, there exists smooth charts containing $p$ and $F(p)$ in which the coordinate representation of $F$ is linear.
        \item $F$ has constant rank.
    \end{enumerate}
\end{corollary}
\begin{proof}
    First we suppose $F$ has linear coordinate representation in a neighborhoods of each point. Since every linear map has constant rank, it follows that $F$ has constant rank on the following neighborhoods and the fact that $M$ is connected implies that the rank of $F$ is constant on every point of $M$. 
    
    Conversely, if $F$ has constant rank, it follows from the previous theorem that there exists following smooth charts such that $F$ has the representation of the form ~\ref{thm:rank}, which is linear.
\end{proof}
\begin{theorem}
    Let $F: M^m \to N^n$ be a smooth map of constant rank between smooth manifolds.
    \begin{enumerate}
        \item If $F$ is surjective, then it is a smooth submersion.
        \item If $F$ is injective, then it is a smooth immersion.
        \item If $F$ is bijective, then it is a diffeomorphism.
    \end{enumerate}
\end{theorem}
\begin{proof}
    (1) Suppose $F$ is surjective and has constant rank $r < n$. Let $p \in M$, by the Constant Rank Theorem, there exists smooth local charts $(U_p,\varphi_p)$ and $(V_{F(p)},\psi_{F(p)})$ corresponding to $p$ and $F(p)$  such that $F(U_p) \subseteq V_{F(p)}$ in which $F$ has a coordinate representation of the form
    \begin{equation}
        \tilde{F_p}(x^1,\dots,x^m) = (x^1,\dots,x^r,0,\dots).
    \end{equation}
    For the sake of condition, we can shrink $U$ into an $r$-dimensional regular coordinate ball and $F(\clo{U_p}) \subseteq V$. Then $F(\clo{U_p})$ is a compact subset of the set \[\{y \in V \mid y^{r +1} = \dots = y^n = 0\}.\]
    Since such a closed $r$-dimensional cannot contain any $n$-dimensional regular coordinate bals in $N$, $F(\clo{U_p})$ does not include any open subset of $N$. Hence it is nowhere dense in $N$. 

    Since charts $\{(U_p,\varphi_p)\}_{p \in M}$ is an open cover of $M$, then we can choose a subcover $\{(U_p,\varphi_p)\}_{p \in A}$ on $M$. Since $F(M)$ is covered by the corresponding subcover $\{F(\clo{U_p})\}_{p \in A}$, which is countable and every element is nowhere dense, by the Baire category theorem, $\clo{F(M)} = \varnothing$, which contradicts the fact that $F$ is surjective, hence it must be a smooth submersion.
     
    (2)  Suppose $F$ is injective and $r < m$, follows from $(7)$, we have 
    \[
    \tilde{F_p}(x^1,\dots,x^m) = \tilde{F_p}(x^1,\dots,x^r,0,\dots)
    \]
    which implies that $U$ is $r$-dimensional, contradiction. Hence $F$ must be a smooth immersion.

    (3) Follows from $(1)$ and $(2)$, $F$ is both smooth submersion and immersion, it thus is local diffeomorphism. The fact that $F$ is bijective implies that it is diffeomorphism from $M$ onto $N$.
\end{proof}
\subsection{Embeddings}
\begin{definition}
    Let $M^m$ and $N^n$ be smooth manifolds. A \text{smooth embedding of $M$ into $N$} is a smooth immersion $F: M \to N$ that yields a homeomorphism onto its image $F(M) \subseteq N$.
\end{definition}
\begin{proposition}
Let $F: M^m \to N^n$ be an injective smooth immersion. If any of the following holds, then $F$ is a smooth embedding.
\begin{enumerate}
    \item $F$ is an open or closed map.
    \item $F$ is a proper map.
    \item $M$ is compact.
    \item $M$ has empty boundary and $\dim M = \dim N$.
\end{enumerate}
\end{proposition}
\begin{proof}
    If $F$ is open or closed map, then the inverse map $F^{-1}: F(M) \to M$ is continuous and thus $F$ is a smooth embedding. If $F$ is proper, then it is closed, and if $M$ is compact, $F$ is also closed. Finally assume $M$ has empty boundary and $\dim M = \dim N$. Since $F$ is both smooth immersion and submersion, then $dF_p$ is nonsingular everywhere, and the problem 4.2 implies that $F(M) \subseteq \inte(N)$. Then $F: M \to \inte{N}$ is locally diffeomorphism, so it is open map. Hence $F: M \to N$ is an embedding.
\end{proof}
\begin{theorem}[Local Embedding Theorem]
    Suppose $M^m$ and $N^n$ are smooth manifolds, and $F: M \to N$ is a smooth map. Then $F$ is a submersion if and only if every point in $M$ has a neighborhood $U \subseteq M$ such that $F|_U: U \to N$ is a smooth embedding.
\end{theorem}
\subsection{Submersions}
\begin{definition}
    Let $\pi:M \to N$ be a continuous map, a \textit{section of $\pi$} is a continuous right inver for $\pi$, which is a continuous map $\sigma: N \to M$ such that $\pi \circ \sigma = \mathrm{Id}_N$. Analogously, a \textit{local section of $\pi$} is a continuous map $\sigma: U \to M$ defined by some open subset $U \subseteq N$ such that $\pi \circ \sigma = \mathrm{Id}_U$.
\end{definition}
\begin{theorem}
    Let $M^m$ and $N^n$ be smooth manifolds, and $\pi: M \to N$ be a smooth map. Then $\pi$ is a smooth submersion if and only if every point of $M$ is in the image of a smooth local section of $\pi$.
\end{theorem}
\begin{theorem}
    Let $M$ and $N$ be smooth manifolds, and suppose $\pi: M \to N$ is a smooth submersion. Then $\pi$ is an open map, and if it is surjective it is a quotient map.
\end{theorem}
\begin{theorem}
    Let $\pi: M^m \to N^n$ be a surjective smooth submersion. For any smooth manifold $P$ with or without boundary, a map $F: N \to P$ is smooth if and only if $F \circ \pi$ is smooth. 
    \[
\begin{tikzcd}
M \arrow[dr, "F\circ \pi"'] \arrow[r, "\pi"] 
& N \arrow[d, "F"] \\
& P
\end{tikzcd}
\]

\end{theorem}
\subsection{Smooth Covering Maps}
\begin{definition}
    Let $\pi: M \to N$ is called a \textit{smooth covering map} if
    \begin{enumerate}
        \item $\pi$ is smooth and surjective.
        \item For abitrary $q \in N$, there exists an open neighborhoods $V$ containing $q$ such that 
        \[
            \pi^{-1}(V) = \bigsqcup_{\alpha} U_{\alpha},
        \]
        where $U_{\alpha} \subseteq M$ are open, disjoint and connected. 
        \item For each $\alpha$, we have 
        \(
            \pi_{U_\alpha}:U_{\alpha}\to V
        \)
        is a diffeomorphism.
    \end{enumerate}
\end{definition}
\begin{proposition}
    Let $E$ and $M$ be nonempty connected smooth manifolds. If $\pi: E \to M$ is a proper local diffeomorphism, then $\pi$ is a smooth covering map.
\end{proposition}

\subsection{Problems}
\begin{problem}
    Suppose $M$ is a smooth manifold (without boundary), $N$ is a smooth manifold with boundary, and $F: M \rightarrow N$ is smooth. Show that if $p \in M$ is a point such that $d F_p$ is nonsingular, then $F(p) \in \operatorname{Int} N$. (Used on pp. 80, 87.)

\end{problem}
\begin{proof}
    Suppose the contrary holds that $F(p) \in \pa{N}$.  Let $(x^i)$ for $i = 1,\dots,m$  and $y^j(x^i)$ and $j = 1,\dots, n$ be the coordinate chart containing $p$ and the boundary chart containing $F(p)$. Since $F(p) \in \pa N$, it follows that $y^n(x^i)(x) = 0$ for all $x \in V$, we thus have 
    \[
        \frac{\pa (y^n)}{\pa x^i}\bigg|_{\varphi(p)} = 0 
    \]
    for all $i = 1,2,\dots,m$. Since we have 
    \[
        dF_p\left(\frac{\pa}{\pa x^i }\bigg|_{p}\right) = \sum_{j = 1}^n\frac{\pa F^j}{\pa x^i}\frac{\pa}{\pa y^j}\bigg|_{F(p)} 
    \]
    Then the last row of $DF_p$ vanishes to zero, which means $dF_p$ is singular, contradiction.
\end{proof}
\begin{problem}
        Formulate and prove a version of the rank theorem for a map of constant rank whose domain is a smooth manifold with boundary. [Hint: after extending $F$ arbitrarily as we did in the proof of Theorem 4.15, follow through the proof of the rank theorem until the point at which the constant-rank hypothesis is used, and then explain how to modify the extended map so that it has constant rank.]
\end{problem}
\begin{problem} We denote $\bb{T}^2 = \bb{S}^1 \times \bb{S}^1$ be the 2-dimensional torus.
     Let $\alpha \in \bb{R}$ be an irrational number and $\gamma: \mathbb{R} \rightarrow \mathbb{T}^2$ be the curve satisfying:
     \[
            \gamma(t) = (e^{2\pi \alpha t},e^{2\pi \alpha i t})
     \] Show that the image set $\gamma(\mathbb{R})$ is dense in $\mathbb{T}^2$. 
\end{problem}
\begin{proof}
    We aim to prove the following lemma:
    \begin{lemma}[Kronecker's Density Theorem]
        Let $\alpha \in \bb{R}$ be an irrational number, then the set 
        \[
            S = \{\{n \alpha\} \mid n \in \bb{Z}\}
        \]
        is dense in $[0,1)$.
    \end{lemma}
    \begin{proof}[Proof of the lemma]
        Observe that every element of $S$ is dinstict since $\{n_1\alpha\} = \{n_2\alpha\}$ implies that 
        \[
            n_1\alpha - \lfloor n_1 \alpha \rfloor =n_2\alpha - \lfloor n_2 \alpha \rfloor \lra (n_1 - n_2)\alpha = \lfloor n_1 \alpha \rfloor - \lfloor n_2 \alpha \rfloor.
        \]
        Then it follows that $\alpha$ is rational, which leads to contradiction.
        Let $m \in N$, divide $(0,1)$ into the following interval:
        \[
            \left[0,\frac{1}{m}\right), \left[\frac{1}{m},\frac{2}{m}\right), \dots, \left[\frac{m - 1}{m},\frac{m}{m}\right)
        \]
        The Dirichlet principle implies that there exists $k \leq m$ and $n_1,n_2 \in \bb{Z}$ such that \[\{n_1\alpha\}, \{n_2\alpha\} \in \left[\frac{k - 1}{m},\frac{k}{m}\right).\]
        For the sake of condition, we can assume that $n_1 < n_2$. Since we have the following expression
        \begin{equation}
            \begin{aligned}
                 \{(n_2 - n_1)\alpha\} &= (n_2 - n_1)\alpha - \lfloor(n_2 - n_1)\alpha\rfloor\\
                &=  \lfloor n_2\alpha\rfloor + \{n_2\alpha\} - \lfloor n_1\alpha\rfloor - \{n_1\alpha\} - \lfloor(n_2 - n_1)\alpha\rfloor
            \end{aligned}
        \end{equation}
        If $\lfloor n_2\alpha\rfloor  = \lfloor n_1\alpha \rfloor+\lfloor(n_2 - n_1)\alpha\rfloor$, we have 
        \begin{equation}\label{thm:case1}
            \begin{aligned}
         \{(n_2 - n_1)\alpha\} &=\lfloor n_2\alpha\rfloor + \{n_2\alpha\} - \lfloor n_1\alpha\rfloor - \{n_1\alpha\} - \lfloor(n_2 - n_1)\alpha\rfloor\\
        &= \{n_2\alpha\} - \{n_1\alpha\} < \frac{k}{m} - \frac{k - 1}{m} = \frac{1}{m}
        \end{aligned}
        \end{equation}
        
        Let $q = \{(n_2 - n_1)\alpha\} < \dfrac{1}{m}$, for each $p \leq m$, we need to find $N \in \bb{N}$ and such that 
        \begin{equation}\label{thm:p}
            \dfrac{p - 1}{m} < Nq < \dfrac{p}{m}
        \end{equation}
            
        
        As we calculate the distance between two side of the inequality
        \[
           \frac{1}{q}\left(\frac{p}{m} - \frac{p - 1}{m}\right) = \frac{1}{qm} > 1
        \]
        Then it is possible to choose an integer $N$ satisfying~\ref{thm:p}. And since $q < \dfrac{1}{m}$, we thus have 
        \[
        \{N(n_2 - n_1)\alpha\} = qN \in \left[\frac{p- 1}{m},\frac{p}{m}\right).
        \]
        If $\lfloor n_2\alpha\rfloor  = \lfloor n_1\alpha \rfloor+\lfloor(n_2 - n_1)\alpha\rfloor + 1$, we also have
                \begin{align*}
         \{(n_2 - n_1)\alpha\} &=\lfloor n_2\alpha\rfloor + \{n_2\alpha\} - \lfloor n_1\alpha\rfloor - \{n_1\alpha\} - \lfloor(n_2 - n_1)\alpha\rfloor\\
        &= \{n_2\alpha\} - \{n_1\alpha\} + 1 \geq \frac{k - 1}{m} - \frac{k}{m} + 1 = \frac{m - 1}{m}
        \end{align*}
        Let $q = \{(n_2 - n_1)\alpha\} > \dfrac{m - 1}{m}$, we wish to scale $q$ by by integer such that $qN =  \{N(n_2 - n_1)\alpha\} < \frac{1}{m}$. Indeed, this motivates us to prove the following approximation lemma:
        \begin{lemma}[Dirichlet's Approximation Theorem]
            Given $\alpha \in \bb{R}$ and any positive integer $N$. Then there exists $m,n \in \bb{N}$ such that 
            \[
                |n\alpha - m| < \frac{1}{N}
            \]
        \end{lemma}
        \begin{proof}[Proof of the Dirichlet's lemma]
            Let $f(x) = \{x\}$. Apply the Dirichlet's principle for $N + 1$ numbers 
            \[
                \{f(i\alpha): i = 0,\dots, N\}
            \]
            lying in 
            \[
                \left[0,\frac{1}{N}\right),\dots, \left[\frac{N -1}{N},1\right)
            \]
            Then there exists $f(i\alpha)$ and $f(j\alpha)$ lies in one of the subintervals above. This implies that 
            \[
                |f(j\alpha) - f(i\alpha)| < \frac{1}{N}
            \]
            Setting $n = j - i$ and $m = \lfloor j \alpha\rfloor -\lfloor i\alpha\rfloor$ and we are done.
        \end{proof}
        Back to the main lemma, since there exists $m, n \in \bb{N}$ such that 
        \[
            |nq - m| < \frac{1}{m}.
        \]
        Then we can choose $N = n$ and then our wishing statement is proven. Since $\{N(n_2 - n_1)\alpha\} \in \left(0,\frac{1}{m}\right]$, we establish the same proof in~\ref{thm:case1}. In general, for any $m \in \bb{N}$ and $k \in \bb{N}$ such that $k \leq m$, the open interval 
        \[
           I^k_m = \left(\frac{k - 1}{m},\frac{k}{m}\right)
        \]
        contains at least an element in $S$. And since the collection $\{I^k_m\}$ defines an open cover of $[0,1)$. Hence every open subset of $[0,1)$ contains at least an element in $S$. $S$ thus is dense in $[0,1)$.
    \end{proof}
    We would rather use the below corollary of the above lemma.
    \begin{corollary}
        Let $\alpha \in \bb{R}$ be an irrational number, then the set
        \[
            S = \{m + n \alpha \mid m,n \in \bb{Z}\}
        \]
        is dense in $\bb{R}$
    \end{corollary}
    Back to the problem, let $\alpha(x,y) = (e^{2\pi i x}, e^{2 \pi i y}) \in \bb{T}^2$ be abitrary. Using the notation $\exp(x)$ for $e^{2\pi i x}$, we consider the following substraction
    \begin{equation}
        \begin{aligned}
            \norm{\gamma(t) - \alpha(x,y)} &= \norm{e^{2 \pi i t} - e^{2\pi i x}, e^{2 \pi i\alpha t} - e^{2\pi i y}}\\
        \end{aligned}
    \end{equation}
    Let $m,n \in \bb{N}$ be abitrary and substitute $t = x + n$. Since $2p\pi  \equiv 0 \pmod{2\pi}$ for all $p \in \bb{N}$, we obtain 
    \begin{equation}\label{thm:absolute}
        \begin{aligned}
            \norm{\gamma(t) - \alpha(x,y)} &= \norm{\exp(x + n) - \exp(x), \exp(\alpha x + \alpha n + m) -\exp(y)}\\
            &= \norm{0, \exp(\alpha x + \alpha n + m) -\exp(y)}
        \end{aligned}
    \end{equation}
    Since we have \[|e^{i t_1} - e^{it_2}| \leq |t_1 - t_2|\] for all $t_1, t_2 \in \bb{R}$, using this inequality for~\ref{thm:absolute} yields 
    \begin{equation}\label{thm:36}
        \begin{aligned}
        \norm{\gamma(t) - \alpha(x,y)} &= |\exp(\alpha x + \alpha n + m) -\exp(y)|\\
        &\leq |\alpha x + \alpha n + m - y|\\
        &= |(\alpha n + m)- ( y-\alpha x)|
    \end{aligned}
    \end{equation}
    
    Since $y-\alpha x \in \bb{R}$ and the set 
    \[
        S= \{\alpha n + m \mid n,m \in \bb{N}\}
    \]
    is dense in $\bb{R}$, then one can choose a sequence $(a_n)_{n \in \bb{N}} \subseteq S$ which has the form $a_n = x_n \alpha + y_n$ and $$\nlim a_n =  y-\alpha x.$$
    For each $k$,  set $n = x_k$ and $m = y_k$. From~\ref{thm:36}, we obtain
    \[
        \norm{\gamma(t) - \alpha(x,y)} \leq |a_n - (y -\alpha x)| \to 0
    \]
    Therefore, for all $x,y \in \bb{R}$, since we can already choose $t_n \in \bb{R}$ such that $\norm{\gamma(t) - \alpha(x,y)} \to 0$, then $\gamma(\bb{R})$ must be dense in $\bb{T}^2$.
\end{proof}

\begin{problem}
    Let $\mathbb{C P}^n$ denote the $n$-dimensional complex projective space, as defined in Problem 1-9.
(Used on pp. 172, 560.)
    \begin{enumerate}
        \item Show that the quotient map $\pi: \mathbb{C}^{n+1} \backslash\{0\} \rightarrow \mathbb{C} \mathbb{P}^n$ is a surjective smooth submersion.
        \item Show that $\mathbb{C} \mathbb{P}^1$ is diffeomorphic to $\mathbb{S}^2$.
    \end{enumerate}
\end{problem}
\begin{proof}
    
\end{proof}
\begin{problem}
     Let $M$ be a nonempty smooth compact manifold. Show that there is no smooth submersion $F: M \rightarrow \mathbb{R}^k$ for any $k>0$.
\end{problem}
\begin{proof}
    Assume the contrary holds that there exists a smooth submersion $F: M \to \bb{R}^k$ for some $k > 0$. Since $F$ is continuous, then $F(M)$ is compact, hence $F(M)$ is bounded and closed by the Heine-Borel theorem. Applying Constant Rank theorem, there exists local coordinate charts such that one can write $F$ as the following coordinate representation
    \[
        \ti{F}(x^1,\dots,x^m) = (x^1,\dots,x^n)
    \]
    which is an open map, then $F$ is also an open map. Hence $F(M)$ is open, which contradicts the fact that  there is no open compact subset in $\bb{R}^k$.
\end{proof}
\begin{problem}
    Suppose $M$ and $N$ are smooth manifolds, and $\pi: M \rightarrow N$ is a surjective smooth submersion. Show that there is no other smooth manifold structure on $N$ that satisfies the conclusion of Theorem 4.29; in other words, assuming that $\widetilde{N}$ represents the same set as $N$ with a possibly different topology and smooth structure, and that for every smooth manifold $P$ with or without boundary, a map $F: \tilde{N} \rightarrow P$ is smooth if and only if $F \circ \pi$ is smooth, show that $\mathrm{Id}_N$ is a diffeomorphism between $N$ and $\tilde{N}$. [Remark: this shows that the property described in Theorem 4.29 is "characteristic" in the same sense as that in which Theorem A.27(a) is characteristic of the quotient topology.]
\end{problem}
\begin{problem}
    This problem shows that the converse of Theorem 4.29 is false. Let $\pi: \mathbb{R}^2 \rightarrow \mathbb{R}$ be defined by $\pi(x, y)=x y$. Show that $\pi$ is surjective and smooth, and for each smooth manifold $P$, a map $F: \mathbb{R} \rightarrow P$ is smooth if and only if $F \circ \pi$ is smooth; but $\pi$ is not a smooth submersion.
\end{problem}

\begin{problem}
    Let $M$ be a connected smooth manifold, and let $\pi: E \rightarrow M$ be a topological covering map. Complete the proof of Proposition 4.40 by showing that there is only one smooth structure on $E$ such that $\pi$ is a smooth covering map. [Hint: use the existence of smooth local sections.]
\end{problem}
\begin{problem}
    Show that the map $q: \mathbb{S}^n \rightarrow \mathbb{R} \mathbb{P}^n$ defined in Example 2.13(f) is a smooth covering map. (Used on p. 550.)
\end{problem}
\begin{problem}
    Show that a topological covering map is proper if and only if its fibers are finite, and therefore the converse of Proposition 4.46 is false.
\end{problem}
\begin{problem}
    Using the covering map $\varepsilon^2: \mathbb{R}^2 \rightarrow \mathbb{T}^2$ (see Example 4.35), show that the immersion $X: \mathbb{R}^2 \rightarrow \mathbb{R}^3$ defined in Example 4.2(d) descends to a smooth embedding of $\mathbb{T}^2$ into $\mathbb{R}^3$. Specifically, show that $X$ passes to the quotient to define a smooth map $\tilde{X}: \mathbb{T}^2 \rightarrow \mathbb{R}^3$, and then show that $\tilde{X}$ is a smooth embedding whose image is the given surface of revolution.
\end{problem}
\begin{problem}
    Define a map $F: \mathbb{S}^2 \rightarrow \mathbb{R}^4$ by $F(x, y, z)=\left(x^2-y^2, x y, x z, y z\right)$. Using the smooth covering map of Example 2.13(f) and Problem 4-10, show that $F$ descends to a smooth embedding of $\mathbb{R} \mathbb{P}^2$ into $\mathbb{R}^4$.
\end{problem}