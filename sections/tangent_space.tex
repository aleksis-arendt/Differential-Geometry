\chapter{Tangent Space}

\section{Derivatives in Multivariable Calculus}
\begin{definition}
    Let $E$ be a subset of $\bb{R}^n, f: E \to \bb{R}^m$ be a function, $x_0 \in E$ be a point, and let $L: \bb{R}^n \to \bb{R}^m$ be a linear transformation. We say that $f$ is differentiable at $x_0$ with derivative $L$ if we have 
    \[
    \lim_{x \to x_0, x \in E \backslash \{x_0\}} \frac{\norm{f(x) - f(x_0) - L(x - x_0)}}{\norm{x - x_0}} = 0
    \]
\end{definition}
\begin{proposition}
    Suppose $f$ is differentiable at $x_0$ with derivative $L_1$, and also differentiable  at $x_0$ with derivative $L_2$. Then $L_1 = L_2$.
\end{proposition}
\begin{proof}
    Since $f$ is differentiable at $x_0$ with derivative $L_1$ and $L_2$, we have 
    \begin{align*}
                \limit_{x \to x_0} \frac{\norm{L_1(x - x_0) - L_2(x - x_0)}}{\norm{x - x_0}} \leq \limit_{x \to x_0} \frac{\norm{f(x) - f(x_0) - L_2(x - x_0)}}{\norm{x - x_0}} +\\ \limit_{x \to x_0} \frac{\norm{L_1(x - x_0) - (f(x) - f(x_0))}}{\norm{x - x_0}}\\
                = 0
    \end{align*}
    Let $h = x - x_0$, one obtain that 
    \[
        \limit_{h \to 0} \frac{\norm{L_1(h) - L_2(h)}}{\norm{h}} = 0
    \]
    Given $x \in E$ and $t$ be a scalar such that $t \to 0$, then it follows that $tx \to 0$. Since $L_1$ and $L_2$ are linear map, we have 
    \[
        \norm{L_1(x) - L_2(x)} = \frac{\norm{L_1(tx) - L2(tx)}}{\norm{tx}} \to 0
    \]
    Thus $L_1 = L_2$, we are done.
\end{proof}

\begin{definition}
    Let $E$ be a subset of $\bb{R}^n$, $f: E \to \bb{R}^m$ be a function, let $x_0$ be an interior point of $E$, and let $v$ be a vector in $\bb{R}^n$. If the limit
    \[
        \limit_{t \to 0^+, x_0 + tv \in E} \frac{f(x_0 + tv) - f(x_0)}{t}
    \]
    exsists, we say that $f$ is \textit{differentiable in the direction $v$ at $x_0$}, and we denote the above limit by $D_vf(x_0)$:
    \[
        D_vf(x_0) := \limit_{t \to 0^+}\frac{f(x_0 + tv) - f(x_0)}{t}
    \]
\end{definition}
\begin{proposition}
    If $f$ is differentiable at $x_0$, then $f$ is also differentiable in the direction $v$ at $x_0$, and 
    \[
        D_vf(x_0) = f'(x_0)v
    \]
\end{proposition}
\begin{proof}
    This is trivial if $v = 0$, assume that $v \neq 0$. Since $f$ is differentiable at $a$, then there exists a linear map $f'(a): \bb{R}^n \to \bb{R}^n$ satisfying
    \[
        0 = \lim_{h \to 0}\frac{f(a + h) - f(a) - hf'(a)}{\norm{h}},
    \]
    Replacing $h$ by $tv$ and $t \to 0$ yields
    \[
        0 = \lim_{t \to 0^+}\frac{f(a + tv) - f(a) - tvf'(a)}{t\norm{v}}  =\lim_{t \to 0^-}\frac{f(a + tv) - f(a) - tvf'(a)}{t\norm{v}}  
    \]
    Since $\norm{v} \neq 0$, we thus have 
    \[
    0 =\lim_{t \to 0}\frac{f(a + tv) - f(a) - tvf'(a)}{t}  = D_v|af = f'(a)\cdot v.
    \]

\end{proof}
\begin{definition} Let $E$ be a subset of $\bb{R}^n$, let $f: E \to \bb{R}^m$ be a function let $x_0$ be an interior point of $E$, and let $1 \leq j \leq n$. Then the partial derivative of $f$ respect to the $x_j$ variable at $x_0$, denoted $\de{f}{x_j}(x_j)$, is defined by
\[
\de{f}{x_j}(x_j) = \limit_{t \to 0, x_0 + te_j \in E} \frac{f(x_0 + te_j) - f(x_0)}{t} = \frac{d}{dt}f(x_0 + te_j)_{t = 0}
\]
    
\end{definition}
\begin{theorem}\label{thm:derivation}
    Let $E$ be a subset of $\bb{R}^n$, $f: E \to \bb{R}^m$ be a function, $F$ be a subset of $E$, and $x_0$ be an interior pint of $F$. If all the partial derivatives $\de{f}{x_j}$ exist on $F$ and are continuous at $x_0$, then $f$ is differentiable at $x_0$. Moreover if $v = (v_1,v_2,\dots,v_n) \in \bb{R}^n$, the linear transformation $f'(x_0): \bb{R}^n \to \bb{R}^m$ is defined by 
    \[
        f'(x_0)(v) = \sum_{j = 1}^n v_j \de{f}{x_j}(x_0)
    \]
\end{theorem}
\begin{proof}
    We first suppose $L: \bb{R}^n \to \bb{R}^m$ be the linear transformation such that if $v$ is a vector of $\bb{R}^n$ and $v = (v_1,v_2,\dots,v_n)$, then
    \[
        L(v) = \sum_{j = 1}^n v_j \de{f}{x_j}(x_0)
    \]
    We attempt to prove that $L = f'(x_0)$, in other words, this is equivalent to 
    \[
        \limit_{h \to 0}\frac{\norm{f(x_0 + h) - f(x_0) - L(h)}}{\norm{h}}.
    \]
    Indeed, given $\ep > 0$, it suffices to find $\delta > 0$ such that  whenever $h \in B(0,\delta) \backslash \{0\}$, we should have 
    \[
        \norm{f(x_0 + h) - f(x) - L(h)} < \ep\norm{h}.
    \]
    For every $1 \leq j \leq n$, since the partial derivative $\de{f}{x_j}$ exists, the function $g_j(x) = \de{f}{x_j}(x)$ is also continuous in at $x_0$. Thus we can find $\delta_j > 0$ such that whenever $h \in B(0,\delta_i) \backslash \{0\}$, we have 
    $\norm{g_j(x) - g_j(x_0)} < \frac{\ep}{mn}$. Let $\delta = \min\{\delta_1,\dots,\delta_n\}$ and fix $h \in  B(0,\delta) \backslash \{0\}$, it is possible to determine the scalars $(h_1,h_2,\dots,h_n) \in \bb{R}^n$ such that
    \[
        h = h_1e_1 + h_2e_2 + \dots + h_ne_n,
    \]
    where $\{e_1,e_2,\dots, e_n\}$ is standard ordered basis of $\bb{R}^n$. Indeed, by writing $f$ in components as $f = (f_1,\dots,f_m)$, it would follow that 
    \[
        \de{f}{x_j}(x_0) = \left(\de{f_1}{x_j}(x_0), \de{f_2}{x_j}(x_0),\dots, \de{f_n}{x_j}(x_0)\right) \tag{1}.
    \]
    Let $1 \leq i \leq n$, by the Mean Value Theorem, one may locate $t_1$ between $0$ and $h_1$ such that \[f_i(x_0 + h_1e_1) - f_i(x_0) = \de{f_i}{x_1}(x_0 + t_1e_1)h_1.\]
    In particular, since 
    \[
        \norm{\de{f_i}{x_1}(x_0 + t_1e_1) - \de{f_i}{x_1}(x_0)} \leq \norm{\de{f}{x_1}(x_0) - \de{f}{x_1}(x_0)} < \frac{\ep}{mn}
    \]
    Multiplying both sides by $\norm{h_1}$ and replacing the left-side term with $(1)$ obtains that 
    \[
        \norm{f_i(x_0 + h_1e_1) - f_i(x_0) - \de{f_i}{x_1}(x_0)} < \frac{\ep}{mn}\norm{h_1} \leq \frac{\ep}{mn}\norm{h}
    \]
    whenever $1 \leq i \leq n$. Adding $n$ above inequalities for specific $f_1,f_2,\dots,f_n$ and using triangle inequality that $\norm{(y_1,\dots,y_m)} \leq \norm{y_1} + \dots + \norm{y_m}$ yields
    \[  
    \norm{f(x_0 + h_1e_1) - f(x_0) - \de{f}{x_1}(x_0)} < \frac{\ep\norm{h}}{n}.
    \]
    Similarly, by developing the analogous process with $x_0 + h_1e_1$ and $x_0 + h_1e_1 + h_2e_2$, we thus have 
    \[
     \norm{f(x_0 + h_1e_1 + h_2e_2) - f(x_0 + h_1e_1) - \de{f}{x_1}(x_0)} < \frac{\ep\norm{h}}{n}.
    \]
    whenever $h \in B(0,\delta)\backslash \{0\}$. More generally, for each $0 \leq k\leq n - 1$, one can obtain 
    \[
       \norm{u_k} = \norm{ f\left(x_0 + \sum_{j = 1}^{k + 1} h_je_j\right)  - f\left(x_0 + \sum_{j = 1}^{k} h_je_j\right) - \de{f}{x_k}(x_0)} < \frac{\ep\norm{h}}{n}.
    \]
    Summing all these terms together implies
    \begin{align*}
        \norm{u_1 + u_2 + \dots + u_n} &= \norm{f(x_0 + h_1e_1 + \dots + h_ne_n) - f(x_0) - \sum_{j = 1}^n h_j\de{f}{x_j}(x_0)}\\
        &= \norm{f(x_0 + h) - f(x_0) - L(h)}\\
        &\leq \norm{u_1} + \norm{u_2} + \dots + \norm{u_n}\\
        &< \ep \norm{h}
    \end{align*}
    Therefore, $L$ is derivative of $f$ at $x_0$, as desired.
    
\end{proof}
\begin{definition}
    If $f: E \to \bb{R}$ is a real-valued function, and we define the gradient $d f(x_0)$ of $f$ at $x_0$ to be the $n$-dimensional row vector 
    \[
        d f(x_0):= \left(\de{f}{x_1}(x_0), \de{f}{x_2}(x_0), \dots, \de{f}{x_n}(x_0)\right)
    \]
\end{definition}
\section{Tangent Vectors of Euclidean Space}

\begin{definition}
   A derivation at $p \in \bb{R}^n$ is a linear map $\ome: C^{\infty}(\bb{R}^n) \to \bb{R}$ satisfying the Leibniz product rule: 
    \[
        \ome(fg) = f\ome(g) + g\ome(f) \text{ for all smooth maps }g,f \in C^{\infty}(M)
    \]
    We denote the set $T_p\bb{R}^n$ to be the set containing all kind of following derivation, that is 
    \[
        T_p \bb{R}^n := \{\ome: C^{\infty}(M) \to \bb{R} \mid \ome \text{ linear and Leibniz}\}.
    \]  
\end{definition}
\begin{proposition}
    Let $a \in \bb{R}^n$, for each geometric tangent vector $v_a \in \bb{R}$, the map $d \cdot v_a: C^{\infty}(\bb{R}^n)\to \bb{R}$ is a derivation at $a$. Moreover, the operation $v_a \mapsto d(a) \cdot v_a$ is an isomorphism from $\bb{R}^n_a$ onto $T_a\bb{R}^n$.
\end{proposition}
\begin{proof}
    The fact $d \cdot v_a$ is a consequence of the linearity and Lebniz implies that it is a derivation. It suffices to prove that $\mc{L}: v_a \mapsto d \cdot v_a$ is an isomorphism. 

    For injectivity, suppose $\mc{L}(v_a) = \mc{L}(w_a)$ for some geometric tangent vector $v_a, w_a$, it folows that 
    \begin{align*}
        0 &= \mc{L}(v_a - w_a)(x^j) = d \cdot (v_a - w_a)(x^j) =  \left(\frac{\partial}{\partial x^1}(x^j), \dots, \frac{\partial}{\partial x^n}(x^j)\right)\left(v_a - w_a\right)\\
        &= \frac{\pa}{\pa x_i}(x^j)(v^i -w^i) = \delta_{ij}(v^i - w^i) = v^j - w^j,
    \end{align*}
    for all $j = 1, \dots, n$. Thus $\mc{L}$ is injective. To prove surjectivity, let $w \in T_a\bb{R}^n$ be abitrary and let $v$ the the tangent vector of $\bb{R}^n_a$ such that $w(x^i)= v^i$, it suffices to prove that $w = d\cdot v_a$. By the Taylor's theorem, one can write 
    \[
        f(x) = f(a) + \left[\frac{\pa f}{\pa x^i}(a)(x^i - a^i)\right] + \left[(x^i - a^i)(x^j - a^j)\int_{0}^1(1 - t)\frac{\pa^2f}{\pa x^i x^j}(a + t(x-a))dt\right]
    \]
    Differentiating both sides by $w$ yields 
    \[
        w(f(x)) = w(f(a)) + \frac{\pa f}{\pa x^i}(a)(w(x^i) - a^i) = \frac{\pa f}{\pa x^i}(a)v^i = d \cdot v_a (a)
    \]
    Hence, $\bb{R}^n_a$ is isomorphic to $T_a\bb{R}^n$ by the following operation.
\end{proof}

\begin{definition}[Tangent Vectors on Manifolds]
    Let $M$ be a smooth manifold, a derivation at $p \in M$ is a linear map $\ome: C^{\infty}(M) \to \bb{R}$ satisfying the Leibniz product rule: 
    \[
        \ome(fg) = f\ome(g) + g\ome(f) \text{ for all smooth maps }g,f \in C^{\infty}(M)
    \]
    We denote the set $T_p(M)$ to be the set containing all kind of following derivation, that is 
    \[
        T_pM := \{\ome: C^{\infty}(M) \to \bb{R} \mid \ome \text{ linear and Leibniz}\}.
    \]
\end{definition}
\begin{definition}[The Differential of a Smooth Map]
    Let $F: M \to N$ be the smooth map between smooth manifolds, for each $p \in M$, we define the map 
    \[
        d F_p: T_pM \to T_{F(p)}N,
    \]
    called the \textit{differential of $F$ at $p$} that acts on $f \in C^\infty(N)$ by the rule 
    \[
        d F_p(v)(f) = v(f\circ F).
    \]
\end{definition}
\begin{proposition}
    The operation $d F_p$ defined above is a derivation.
\end{proposition}
\begin{proposition}
    Let $M$ be a smooth manifold, $p \in M$ and $v \in T_pM$. If $f,g \in C^{\infty}(M)$ agree on some neighborhood of $p$, then $vf = vg$.
\end{proposition}
\begin{proposition}
    If $M$ is an $n$-dimensional smooth manifold, then for each $p\in M$, the tangent space $T_pM$ is an $n$-dimensional vector space.
\end{proposition}

\begin{definition}[Second definition of Tangent Space] Let $M$ be a smooth manifold and $p \in M$. We say every the smooth function $\zeta: (-\ep,+\ep) \to M$ such that $\zeta(0) = p$ is \textit{a $p$-path}. Two $p$-path $\alpha$ and $\beta$ is said to satisfies the $\sim$ equivalent relation if 
    \[
        \frac{d}{dt}(f(\alpha(t)))\bigg|_{t = 0}=  \frac{d}{dt}(f(\beta(t)))\bigg|_{t = 0}
    \]
    for all smooth map $f\in C^{\infty}(M)$. Then the tangent space at $p$ is defined as:
    \[
        T_p(M) := \{\left[\zeta'(0)\right]| \text{ Smooth curve }\zeta: (-\ep,+\ep) \to M, \zeta(0) = p \}
    \]
\end{definition}

\begin{proposition}
    The tangent spaces at $p$ in first and second definition are naturally isomorphic.
\end{proposition}
\begin{proposition}
    If $M$ is a smooth manifolds with dimension $n$ and let $(x_1,x_2,\dots, x_n)$ be a smooth local chart around $p \in M$, then the set 
    \[
        \left\{\frac{\pa}{\pa x_1}\bigg|_p,\frac{\pa}{\pa x_2}\bigg|_p,\dots, \frac{\pa}{\pa x_n}\bigg|_p\right\}
    \]
    forms a basis for $T_pM$.
\end{proposition}
For convenience, if $v \in T_pM$ we write $v = (dx_1,dx_2,\dots,dx_n)$.
\begin{definition}[Tangent Bundle]
    Let $M$ be a smooth manifold, we define the \textit{tangent bundle of $M$} to be the disjoint union of the tangent spaces at all points on $M$ 
    \[
        TM = \coprod_{p \in M} T_pM.
    \]
\end{definition}
\begin{proposition}
    For any $n$-dimensional smooth manifold $M$, the tangent bundle $TM$ has a natural topology and a smooth structure that make it into a $2n$-dimensional smooth manifold. With respect to this structure, the projection $\pi: TM \to M$ is smooth.
\end{proposition}
\begin{proof}
    We first prove that $\pi$ is smooth. Consider the local coordinate chart $(U,\varphi)$ for $M$ which has the form $(x^1,\dots,x^n)$. Notice that $\pi^{-1}(U)$ is an open set in $TM$ and $$\pi^{-1}(U) = TU = \displaystyle \coprod_{p \in U} T_pM,$$ which motivates us to consider the local chart $\ti{\varphi}: \pi^{-1}(U) \to \bb{R}^{2n}$ satisfying 
    \[
        \ti{\varphi}\left(v^i\frac{\pa}{\pa x^i}(p)\right) = (x^i(p),v^i) \text{ and }\ti{\varphi}^{-1}(x^i, v^i) = v^i \frac{\pa}{\pa x^i}(\varphi^{-1}(x))
    \]
    for all $v = v^i\dfrac{\pa}{\pa x^i} \in T_pM$ and $p \in U$. Let $(V,\psi)$ and $(\pi^{-1}(V), \ti{\psi})$ be the similarly defined smooth local chart on $M$ and $TM$, respectively. Since $\varphi$ and $\psi$ is a homeomorphism, they are smoothly compatible. Then it suffices to verify the compatibility of the transition map $\ti{\psi}\circ \ti{\varphi}^{-1}: \ti{\varphi}({\pi^{-1}(U) \cap \pi^{-1}(V)}) \to \ti{\psi}({\pi^{-1}(U) \cap \pi^{-1}(V)})$, which can be computed sufficiently 
    \[
        \ti{\psi}\circ \ti{\varphi}^{-1}( x^i(p),v^i) = \ti{\psi}\left(v^i \frac{\pa}{\pa x^i}(p)\right) = (\ti{x}^i(p),\ti{v}^i )
    \]
    which is clearly a smooth function. Since $M$ is second-countable, one can choose a countable cover $\{U_i\}$ for $M$ and a smooth structure $\{(U_i,\varphi_i)\}$ defined above. Hence we obtain a countable cover $\{\pi{U_i}\}$ for $TM$ which implies that $TM$ is locally Euclidean and second-countable. The Hausdorff property is trival since two points in the same fiber can be seperated by the same chart and in different fiber can be mapped through $\pi$ onto $M$, which is Hausdorff. Hence $TM$ is a manifold, and the corresponding smooth structure $\{(\pi^{-1}(U_i),\ti{\varphi}_i)\}$ implies that $TM$ is smooth manifold. Finally, to see that $\pi$ is smooth, with respect to the charts $(U,\varphi)$ and $(\pi^{-1}(U),\ti{\varphi})$, the coordinate representation is $\pi(x,v) = x$. Hence we are done.  
\end{proof}
\section{Problems}
\begin{problem}\label{thm:3.1}
     Suppose $M$ and $N$ are smooth manifolds with or without boundary, and $F: M \rightarrow N$ is a smooth map. Show that $d F_p: T_p M \rightarrow T_{F(p)} N$ is the zero map for each $p \in M$ if and only if $F$ is constant on each component of $M$.
\end{problem}
\begin{proof}
    Since the converse implication is trival, we only prove the forward one. Let $(x^1,\dots,x^n)$ be a local coordinate chart for a neighborhood $U$ of $p$. Let $v \in T_pM$, then it can be expressed as a linear combination 
    \[
        v = v^i \frac{\pa}{\pa x^i}(p) 
    \]
    Differentiating both sides by $d F_p$ yields 
    \[
        0 = d F_p (v)(x^j) = v^i \frac{\pa (x^j \circ F)}{\pa x^i }(p) = v^i \frac{\pa F^j}{\pa{x^i}}(p) 
    \]
    Since this is true for all $v \in T_pM$, choosing $v = \pa_{j}$ implies that  $\dfrac{\pa F^j}{\pa x^j}(p)$ for all $j$. As a consequence, each component $F_j$ is constant, then $F$ is also constant.
\end{proof}

\begin{problem}
     Let $M_1,\dots, M_k$ be smooth manifolds and for each $j$, let $\pi_j: M_1\times \dots \times M_k \to M_j$ be the projection onto the $M_j$ factor. For any point $p = (p_1,\dots,p_k) \in M_1\times\dots\times M_k$, the map
     \[
        \alpha: T_p(M_1\times\dots\times M_k) \to T_{p_1}M_1 \oplus \dots \oplus T_{p_k}M_k
     \] 
defined by 
\[
    \alpha(v) = (d(\pi_i)_p (v))
\]
is an isomorphism.
\end{problem}
\begin{proof}
    We first verify that $\alpha$ is well-defined. For every $j = 1,\dots, k$, since the differential $$d(\pi_j)_p:  T_p(M_1\times \dots \times M_k)\to T_{\pi_j(p)}M_j= T_{p_j}M_j$$
    sends the product tangent vector $v$ to every seperate tangent space $T_{p_j}M_j$. This ensures that the image of $\alpha$ always lies on the direct sum of following tangent spaces. We now prove that $\alpha$ is injective. Suppose $\alpha(v) = \alpha(w)$, since $\alpha$ is linear due to the linearity of the differentials, we thus have $\alpha(v) - \alpha(w) = \alpha(v - w) = 0$ implies that $d(\pi_i)_p(v - w) = 0$ for all $i = 1 ,\dots, k$. Let $k_i = \dim{M_i}$ for all $i$ and consider the local coordinate $(x^1_{1},\dots,x^{k_1}_1,x^1_{2},\dots, x^{k_k}_{k}) = (x^1,\dots, x^{k_1 + \dots + k_k})$. Let $j = 1,\dots, k$ and $t = 1,\dots, k_j$ be fixed and $i$ varies from $1 $ to $k_1 + \dots + k_k$, expressing the vector $u = v - w$ in coordinate $u = u^i\frac{\pa}{\pa x^i}(p)$ yields
    \[
        0 = d(\pi_j)_p(u)(x^t) = u^i\frac{\pa(x^t \circ \pi_j)}{\pa x^i}(p) = u^i \frac{\pa x^t_j}{\pa x^i} (p) = u^i \frac{\pa x^{k_{t -1} + j}}{\pa x^i}(p) = u^{k_{t - 1} +j}
    \]
    Since $t$ and $j$ was abitrary, then it follows that $u^i = v^i - u^i = 0$, hence $\alpha$ is injective. Let $w = (w_1,\dots, w_k) \in  T_{p_1}M_1 \oplus \dots \oplus T_{p_k}M_k$, and let 
    \[
        v = w^j_i\frac{\pa}{\pa x^{k_{i - 1} + j}}.
    \]
    The computation above implies that $\alpha(v) = w$. Thus $\alpha$ is bijective.
\end{proof}
\begin{problem}
    Prove that if $M$ and $N$ are smooth manifolds, then $T(M \times N)$ is diffeomorphic to $T M \times T N$.
\end{problem}
\begin{problem}
     Show that $T \mathbb{S}^1$ is diffeomorphic to $\mathbb{S}^1 \times \mathbb{R}$.
\end{problem}
\begin{proof}
    We begin to construct by viewing $\bb{S}^1$ in $\bb{C}$. Let $(U,\theta)$ and $(V, \psi)$ be the angle local chart such that $\theta: \bb{S}^1\backslash\{1\} \to (0,2\pi)$ and $\psi: \bb{S}^1\backslash\{-1\} \to (-\pi,\pi)$. We consider the map $F: T\bb{S}^1 \to \bb{S}^1 \times\bb{R}$ satisfying
    \[
        F \left(x, v\pa_{\theta}|_x\right) = (x,v) \text{ on } U \text{ and }
        F\left(x, v\pa_{\psi}|_x\right) = (x,v) \text{ on }V.
    \]
    Since we have $\theta = \psi - \pi$ and $\dfrac{d \psi}{d\theta} = 1$ then $\pa_{\theta}|_x=\pa_{\psi}|_x$ on the restriction to $F$ on $U \cap V$ since , it follows that $F$ is well-defined. To check that $F$ is bijection, define the candidate function $G: \bb{S}^1 \times \bb{R} \to T\bb{S}^1$ satisfying
    \[
G(x,v) =\begin{cases}
     \left(x, v\pa_{\theta}|_x\right), &\quad x \in U,\\
     \left(x, v\pa_{\psi}|_x\right), &\quad x \in V
\end{cases}
    \]
    Since $G$ is well-defined and $F \circ G(x,v) = G \circ F(x,v) = (x,v)$, thus $F$ is bijective.

    In addition, $F$ is smooth on $U$ and $V$ since the coordinate expression $(x,v) \mapsto (x,v)$ is smooth on $U$ and $V$, and agrees smoothly on $U \cap V$. Computing the transition $\theta \circ \psi^{-1}:\psi(U \cap V) \to \theta(U \cap V)$, we obtain
    \[
        \theta \circ \psi^{-1}(\psi(x)) =\theta(x )
    \] 
    which is smooth, the gluing lemma implies that $F$ is smooth globally. As $G$ is smooth by the analogous argument, hence $F$ is diffeomorphism $T \mathbb{S}^1 \cong \mathbb{S}^1 \times \mathbb{R}$.
\end{proof}
\begin{problem}
    Let $\mathbb{S}^1 \subseteq \mathbb{R}^2$ be the unit circle, and let $K \subseteq \mathbb{R}^2$ be the boundary of the square of side 2 centered at the origin: $K=\{(x, y): \max (|x|,|y|)=1\}$. Show that there is a homeomorphism $F: \mathbb{R}^2 \rightarrow \mathbb{R}^2$ such that $F\left(\mathbb{S}^1\right)=K$, but there is no diffeomorphism with the same property. [Hint: let $\gamma$ be a smooth curve whose image lies in $\mathbb{S}^1$, and consider the action of $d F\left(\gamma^{\prime}(t)\right)$ on the coordinate functions $x$ and $y$.] (Used on p. 123.)
\end{problem}
\begin{proof}
    Suppose that there is a diffeomorphism $F: \bb{R}^2 \to \bb{R}^2$ such that $F(\circl^1) = K$. Following to the hint, let $\gamma: [0,2\pi) \to \bb{S}^1$ be the smooth curve satisfying $\gamma(t) = (\cos(t),\sin(t))$. Since $\gamma$ is a homeomorphism and has smooth inverse, which is 
    \[
    \gamma^{-1}(x,y) = \gamma^{-1}(x,\pm \sqrt{1 - x^2}) = \arctan\left(\frac{\pm \sqrt{1 - x^2}}{ x}\right)
    \]
    Then $\gamma$ is also a diffeomorphism, thus the composition $F \circ \gamma: [0,2\pi) \to K$ is also a diffeomorphism. Since $F \circ \gamma$ is onto, then there exists $t_0$ such that $F \circ \gamma(t_0) = (1,1)$. Let $I$ be an open interval containing $(F \circ \gamma)^{-1}(1,1)$, then one can split $I$ such that $F \circ \gamma (I_x) \subseteq (-1,1)\times \{1\}$
    and $F \circ \gamma (I_y) \subseteq \{1\}\times \subseteq (-1,1)$. Consequently, we have 
    \begin{equation}\label{eq:2}
        F \circ \gamma(t) = \begin{cases}
            (x \circ F \circ \gamma(t), 1) &\text{ if }t \in I_x\\
            (1,y \circ F \circ \gamma(t)) &\text{ if }t \in I_y\\
        \end{cases}
    \end{equation}
    Since the velocity of $F \circ \gamma$ at $t_0$ is 
    \[
    \left(\dfrac{d(x \circ F \circ \gamma)}{dt}(t_0),0\right) \text{ and } \left(0,\dfrac{d(y \circ F \circ \gamma)}{dt}(t_0)\right)
    \] by ~\eqref{eq:2}, respectively, then two expressions for $(F \circ \gamma)(t_0)$ must concide, which means $(F \circ \gamma)'(t_0) =0$, it follows that $dF(\gamma'(t_0)) = (F \circ \gamma)'(t_0) = 0$, which leads to contradiction since $dF_{\gamma'(t_0)}$ is globally homeomorphism and $\gamma'(t_0) \neq 0$, hence no such diffeomorphism $F$ can exists.
\end{proof}
\begin{problem}
    Consider $\mathbb{S}^3$ as the unit sphere in $\mathbb{C}^2$ under the usual identification $\mathbb{C}^2 \leftrightarrow \mathbb{R}^4$. For each $z=\left(z^1, z^2\right) \in \mathbb{S}^3$, define a curve $\gamma_z: \mathbb{R} \rightarrow \mathbb{S}^3$ by $\gamma_z(t)=$ ($e^{i t} z^1, e^{i t} z^2$). Show that $\gamma_z$ is a smooth curve whose velocity is never zero.
\end{problem}
\begin{proof}
    Let $z^1 = x^1 + iy^1 $and $z^2 = x^2 + iy^2$, then we can write $\gamma_z$ under the usual identification as 
    \begin{equation}
        \begin{aligned}
            \gamma_z(t) &= (e^it z^1, e^it z^2) = \left((\cos(t) + i\sin(t))(x^1 + iy^1),(\cos(t) + i\sin(t))(x^2 + iy^2)\right)\\
            &= (x^1\cos(t) - y^1\sin(t) + i(x^1\sin(t) + y^1\cos(t)),x^2\cos(t) - y^2\sin(t) + i(x^2\sin(t) + y^2\cos(t)))\\
            &\lra (x^1\cos(t) - y^1\sin(t),x^1\sin(t) + y^1\cos(t),x^2\cos(t) - y^2\sin(t),x^2\sin(t) + y^2\cos(t) )
        \end{aligned}
    \end{equation}
    Let $\alpha(t) = y^1\sin(t) + x^1\cos(t) $ and $\beta(t) = y^2\sin(t) + x^2\cos(t)$, one can rewrite 
    \begin{equation}
        \gamma_z(t) = (\alpha'(t),\alpha(t), \beta'(t),\beta(t))
    \end{equation}
    Since the components $\alpha$ and $\beta$ are smooth, it follows that $\gamma_z(t)$ is smooth. Consider the local chart $(U_1^+,\varphi_1^+)$ for $\bb{S}^3$, where $U^+_1 = \{(x^1,\dots,x^4) \in \circl^3 \mid x^1 > 0\}$ and $\varphi_1^+$ satisfies 
    \[
        \varphi_1^+(x_1,x_2,x_3,x_4) = (x_2,x_3,x_4)
    \]
    Then the coordinate representation respect to this chart is $\varphi_1^+ \circ \gamma: \gamma_z^{-1}(U^+_1) \to \bb{R}^3$ and $\varphi_1^+\circ \gamma_z(t) = (\alpha(t), \beta'(t),\beta(t))$. Then the velocity of $\varphi_1^+ \circ\gamma_z$ is 
    \[
        (\varphi_1^+ \circ \gamma_z)'(t) = (\alpha'(t),\beta''(t),\beta'(t)) = (\alpha'(t),-\beta(t),\beta'(t))
    \]
    Since $\alpha(t)' > 0$, then its velocity is nonzero for all $t \in \gamma_z^{-1}(U^+_1)$. An analogous computation for the chart $(U_1^-,\varphi_1^-),(U_2^{\pm},\varphi_2^{\pm}),\dots, (U_4^{\pm},\varphi_4^{\pm})$, since the domain of the collection $\{(U_i^{\pm},\varphi_i^{\pm})\}$ covers $\bb{S}^3$, it follows that the velocity of the smooth curve $\gamma_z$ is never zero.
\end{proof}
\begin{problem}
    Let $M$ be a smooth manifold with or without boundary and $p \in M$. Let $\mathcal{V}_p M$ denote the set of equivalence classes of smooth curves starting at $p$ under the relation $\gamma_1 \sim \gamma_2$ if $\left(f \circ \gamma_1\right)^{\prime}(0)=\left(f \circ \gamma_2\right)^{\prime}(0)$ for every smooth real-valued function $f$ defined in a neighborhood of $p$. Show that the map $\Psi: \mathcal{V}_p M \rightarrow T_p M$ defined by $\Psi[\gamma]=\gamma^{\prime}(0)$ is well defined and bijective. (Used on p. 72.)
\end{problem}