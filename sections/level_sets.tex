\chapter{Level Sets Theorem}
\section{Embedded Submanifolds}
\begin{definition}
    Let $M^m$ be a smooth manifold, an \textit{embedded submanifold of $M$} is a subset $S \subseteq M$ that is a manifold in the subspace topology, endowed with a smooth structure with respect to which the inclusion $S \to M$ is a smooth embedding.
\end{definition}
\begin{definition}
    If $S$ is an embedded submanifold of $M$, the difference $\dim M - \dim S$ is called the \textit{codimension of $S$ in $M$}. An \textit{embedding hypersurface} is an embedded submanifold of codimension $1$.
\end{definition}
\begin{proposition}
    Suppose $M^m$ is a smooth manifold. Then embedded submanifolds of codimension $0$ in M are exactly open submanifolds.
\end{proposition}

\begin{proposition}
    Let $M$ and $N$ be smooth manifolds, and $F: M \to N$ be a smooth embedding. Let $S = F(N)$. With the subspace topology, $S$ is a topological manifold, and it has a unique smooth structure making it into an embedded submanifold of $M$ with the property that $F$ is a diffeomorphism onto its image.
\end{proposition}
\begin{proof}
    If we let $S$ inherit subspace topology from $M$, then $S$ is a topological manifold. Since $F$ is a homeomorphism onto its image, for any chart $(U,\varphi)$ in $N$, then we can choose the corresponding chart $(F(U),\varphi \circ F^{-1})$ for $S$. Since every different choosed chart in $S$ is compatible, it follows that $S$ attains a smooth structure. Then $F$ is local diffeomorphism, the Gluing lemma follows that $F$ is diffeomorphism onto its image, hence $S$ has a unique smooth structure sent from $N$. Let $i: S \hookrightarrow M$ be the inclusion. Since $i = F \circ F^{-1}$ which is a composition of two smooth embedding, then $i$ is also a smooth embedding. Hence $S$ is a embedded submanifold of $M$.
\end{proof}
\begin{proposition}
    Suppose $M$ and $N$ are smooth manifolds. For each $p \in N$, the slice $M \times \{p\}$ is an embedded submanifold of $M \times N$ diffeomorphic to $M$.
\end{proposition}
\begin{proof}
    Since $M \times \{p\}$ is an image of the smooth embedding $x \mapsto (x,p)$, it follows from the above proposition that $M \times \{p\}$ is an embedded submanifold.
\end{proof}
\begin{proposition}
    Suppose $M$ is a smooth $m$-manifold without boundary. $N$ is a smooth $n$ manifold, $U \subseteq M$ is open, and $f:U \to N$ is a smooth map. Let $\Gamma(f)\subseteq M \times N$ denote the graph of $f$
    \[
        \Gamma(f) = \{(x,y) \in M \times N \mid x \in U, y = f(x)\}.
    \]
    Then $\Gamma(f)$ is an embedded $m$-dimensional submanifold of $M \times N$.
\end{proposition}
\begin{proof}
    Let $\gamma: U \to M \times N$ be the map satisfying $\gamma(x) = (x,f(x))$. Since the projection $\pi_M: M \times N \to M$ mapping $(x,y) \mapsto x$ which is a surjective submersion. Then composition $\pi_M \circ \gamma = i: U \hookrightarrow U$ is a smooth map. Hence $\gamma$ is a smooth map by the Characteristic Property Theorem and is injective.

    In addition, the differential $d(\pi_M \circ \gamma)_{p}: T_pM \to T_pM$ satisfying
    \[
        d(\pi_M \circ \gamma)_{p} = d(\pi_M)_{\gamma(p)} \circ d(\gamma)_p = \mathrm{Id}_{T_pM}
    \]
    Thus $d(\gamma)_p$ is injective and it follows that $\gamma$ is a smooth immersion. Hence $\gamma$ is a smooth embedding, then its image $\Gamma(f)$ is an embdedded $m$-dimensional submanifold of $M \times N$.
\end{proof}
\begin{proposition}
    Suppose $M$ is a smooth manifold and $S \subseteq M$ is an embdedded submanifold. THen $S$ is properly embedded if and only if it is a closed subset of $M$.
\end{proposition}
\section{Slice Charts}
\begin{definition}
    Let $U \subseteq \bb{R}^n$ be an open subset. A \textit{$k$-slice of U} is any subset $S$ which has the form 
    \[
        S = \{(x^1,\dots,x^n)\in U\mid (x^{k + 1},\dots,x^n) = \text{constant}  \}.
    \]
    Moreover, let $M^m$ be a smooth manifold and $(U,\varphi)$ be a smooth chart on $M$. If $S$ is a subset of $U$ such that $\varphi(S)$ is a $k$-slice of $\varphi(U)$, then we say that $S$ is a $k$-slice of $U$.
\end{definition}
\begin{definition}
    Let $S \subseteq M$, we say that $S$ satisfies \textit{local $k$-slice condition} if each point of $S$ is contained in the domain of a smooth chart $(U,\varphi)$ for $M$ such that $S \cap U$ is a single $k$-slice in $U$.
\end{definition}
\begin{theorem}
    Let $M$ be a smooth $n$-manifold. If $S \subseteq M$ is an embdedded $k$-dimensional submanifold, then $S$ is a local $k$-slice. Conversely, if $S$ is a local $k$-slice, then $S$ is a $k$-dimensional embedded submanifold of $M$ with the subspace topology and smooth structure.
\end{theorem}
\section{Level Sets}
\begin{definition}
    Let $\Phi: M^m \to N^n$ be any map and $c \in N$, we call the set $\Phi^{-1}(c)$ a \textit{level set of $\Phi$}.
\end{definition}
\begin{lemma}
    If $\Phi: M^m \to N^n$ is a continuous map between smooth manifolds, then every level set of $\Phi$ is closed in $M$.
\end{lemma}
\begin{theorem}
    Let $M^m$ and $N^n$ be smooth manifolds, and let $\Phi: M \to N$ be a smooth map with constant rank $r$. Each level set of $\Phi$ is a properly embedded submanifold of codimension $r$ in $M$.
\end{theorem}
\begin{proof}
    Let $c \in N$ be abitrary and $S = \Phi^{-1}(c) \subseteq M$ be the level set of $\Phi$. For every $p \in P$, the Constant Rank theorem follows that there exists smooth local charts $(U,\varphi)$ centered at $p$ and $(V, \psi)$ centered at $\Phi$ such that $\Phi$ has the coordinate representation 
    \[
        \Phi(x^1,\dots, x^m) = (x^1,\dots, x^r,0,\dots,0) = c
    \]
    Then it follows that $x^1,\dots,x^r$ are constant numbers. Hence $S$ is a $m - r$-slice, the above theorem follows that it is an embedded submanifold of codimension $r$ in $M$. Since the singleton $\{c\}$ is closed in $N$, then $S = f^{-1}(c)$ is closed in $M$. Therefore, $S$ is properly embedded. 
\end{proof}
\begin{definition}
    Let $\Phi: M^m \to N^n$ be a smooth map, a point $p \in M$ is said to be a \textit{regular point of $\Phi$} if $d\Phi_p$ is surjective and it is a \textit{critical point of $\Phi$} otherwise.
\end{definition}
\begin{definition}
    A point $c \in N$ is said to be a \textit{regular value of $\Phi$} if every point of $\Phi^{-1}(c)$ regular or $\Phi^{-1}(c) = \varnothing$, and a \textit{critical value of $\Phi$} otherwise. Then the set $\Phi^{-1}(c)$ is said to be a \textit{regular level set} if $c$ is a regular value of $\Phi$.
\end{definition}
\begin{proposition}
    Let $\Phi: M^m \to N^n$ be a smooth map and $c \in N$ be a regular value. Then $\Phi^{-1}(c)$ is a properly embedded submanifold of codimension $n$.
\end{proposition}
\begin{proof}
    Let $U$ be the set containing all points $p \in M$ such that $\rank(d\Phi_p)  = n$, then $U$ is open in $M$. Notice that $\Phi^{-1}(c) \subset U$ and the map $\Phi: U \to N$ is a submersion, then $\Phi^{-1}(c)$ is a properly embedded submanifold of codimension $n$ in $U$. Since the inclusion $\Phi^{-1}(c) \hookrightarrow U \hookrightarrow M$ is a smooth embdedding and $\Phi^{-1}(c)$ is closed by continuity, it follows that $\Phi^{-1}(c)$ is properly embedded in $M$.
\end{proof}
\begin{proposition}
    Let $S$ be a subset of a smooth manifold $M^m$. Then $S$ is an embedded $k$-submanifold of $M$ if and only if every point of $S$ has a neighborhood $U$ in $M$ such that $U \cap S$ is a level set of a smooth submersion $\Phi: U \to \bb{R}^{m - k}$.
\end{proposition}
\begin{definition}
    Let $S \subseteq M$ be an embedded submanifold, a smooth map $\Phi: M \to N$ such that $S$ is a regular level set of $\Phi$ is called a \textit{defining map for $S$}.
\end{definition}
\section{Problems}
\begin{problem}
    Consider the map $\Phi: \mathbb{R}^4 \rightarrow \mathbb{R}^2$ defined by
$$
\Phi(x, y, s, t)=\left(x^2+y, x^2+y^2+s^2+t^2+y\right) .
$$

Show that $(0,1)$ is a regular value of $\Phi$, and that the level set $\Phi^{-1}(0,1)$ is diffeomorphic to $\mathbb{S}^2$.
\end{problem}
\begin{proof}
    Let $p \in \Phi^{-1}(0,1)$, we aim to show that $\rank(d\Phi_p) = 2$ by computing its Jacobian. We have 
    \[
        D\Phi_p = \begin{bmatrix}
            2x & 1 & 0 & 0\\
            2x & 2y + 1 & 2s & 2t
        \end{bmatrix}
    \]
    Since $x^2  + y = 0$, then the case $x = y = 0$ implies that either $t$ or $s$ must be nonzero, and thus two rows of $D\Phi_p$ is linearly independent. The property also follows for the remaining case that $x, y \neq 0$. Therefore, $d\Phi_p$ is surjective follows that $(0,1)$ is a regular value of $\Phi$.

    To prove that $\Phi^{-1}(0,1)$ diffeomorphic to $\bb{S}^2$, consider the map $F: \Phi^{-1}(0,1) \to \bb{S}^2$ defined by 
    \[
        F(x,y,s,t) = \frac{(x,s,t)}{\sqrt{x^2 + s^2 + t^2}}
    \]
    Since $F$ has the inverse function 
    \[
        F^{-1}(x,y,z) = \sqrt{\frac{x^2 - 1 + \sqrt{5x^4 -2x^2 + 1}}{2x^4}}(x,-x^2,y,z)
    \]
    whenever $x \neq 0$ and $F^{-1}(x,y,z) = (x,y,z)$ if $x = 0$, then it follows that $F$ is bijective. Since $\Phi^{-1}(0,1)$ and $\bb{S}^2$ are $2$-dimensional, to prove that $F$ is diffeomorphism, it is enough to show that it has constant rank $2$. View $F$ as a map $F: \bb{R}^3\backslash \{0\} \to \bb{S}^2$ by forgetting the $y$-coordinate. Let $p \in \Phi^{-1}(0,1)$, we need to show that \begin{equation}
    T_{p}\Phi^{-1}(0,1) \cap \ker(dF_p) = \{0\}
    \end{equation} For any $v \in T_p\bb{R}^3$, consider the line curve $L : \bb{R} \to \bb{R}^3$, satisfying $L(t) = p + tv$, then it follows that 
    \begin{align*}
    dF_p(v) &= \frac{d}{dt}(F\circ L)(t)\bigg|_{t = 0} = \frac{d}{dt}\left(\frac{p + tv}{\sqrt{t^2\norm{v} + 2\langle p ,v \rangle t + p^2}}\right)\bigg|_{t = 0}\\
    &= \left(\frac{v\norm{p + tv}^2 - (\norm{v}t + \langle p ,v \rangle )( p + tv)}{\norm{p + tv}^3}\right)\bigg|_{t = 0}\\
    &= \frac{v\norm{p}^2 - \langle p,v\rangle p }{\norm{p}^3} 
    \end{align*}
    Then $v \in \kernel(dF_p)$ if and only if $v$ is parallel to $p$, which implies that $v = (kx,ks,kt)$ for some constant $k$. Now consider the map $G: \bb{R}^3 \to \bb{R}$ defined by $G(x,y,z) = x^4 + y^2 + z^2 - 1$, its Jacobian is 
    \[
        DG(x,y,z) = \begin{bmatrix}
            4x^3 & 2s & 2t
        \end{bmatrix}
    \]
    Since $G$ is a defining map on $\Phi^{-1}(0,1)$, then it follows that $T_p\Phi^{-1}(0,1) = \ker(dG_p)$, in other words, that is $dG_p(v) = 0$ or 
    \[
        dG_p(v) = \begin{bmatrix}
            4x^3 & 2s & 2t
        \end{bmatrix}v = k(4x^4 + 2s^2 + 2t^2) = 0
    \]
    This is true if and only if $k = 0$, follows that $(5.1)$ is true. Then $dF_p$ is injective on $\Phi^{-1}(0,1)$, thus $G$ has constant rank $2$, which implies that it is a diffeomorphism.
\end{proof}
\begin{problem}
    If $M^n$ is a smooth manifold with boundary, then with the subspace topology, $\pa M$ is a topological $(n - 1)$-manifold (without boundary), and has a smooth structure such that it is a properly embedded submanifold of $M$.
\end{problem}
\begin{proof}
    Let $i: \pa M \to M$ be the natural inclusion, it suffices to prove it is an embedding. Let $p \in \pa M$, then there is a smooth local boundary chart $(U,\varphi) $ centered at $p$, then $\varphi(U)$ to an open subset of $\bb{H}^n$ with the boundary condition. We define another inclusion $j: \pa\bb{H}^n \to \bb{H}^n$, which is a smooth map, then $i$ can be viewed through the composition
    \[
        i|_{U \cap \pa M} = \varphi^{-1}\circ j \circ \varphi: U \cap \pa M \to U 
    \]
    Since the composition of smooth map are smooth, this implies that $i$ is smooth locally in charts, thus it is smooth. Since $i$ is injective and has constant rank $n - 1$, it is immersion. Moreover, since $\pa{M}$ inherits subspace topology from $M$, then the inclusion $i$ must be a homeomorphism onto its image, and since $\pa M$ is closed, $i$ is proper. Thus $i$ is a smooth proper embedding. Hence, $\pa M$ is a topological $(n - 1)$-manifold, it has a smooth structure defined above and is a properly embedded submanifold in $M$.
\end{proof}
\begin{problem}
    Prove Proposition 5.21 (sufficient conditions for immersed submanifolds to be embedded).
\end{problem}
\begin{problem}
    Show that the image of the curve $\beta:(-\pi, \pi) \rightarrow \mathbb{R}^2$ of Example 4.19 is not an embedded submanifold of $\mathbb{R}^2$. [Be careful: this is not the same as showing that $\beta$ is not an embedding.]
\end{problem}
\begin{proof}
    Let $S = \beta(-\pi,\pi)$, suppose $S$ is an embedded submanifold of $\mathbb{R}^2$. Then $S$ is either $0,1$ or $2$-dimensional topological manifold. If $S$ is $0$-dimensional, then the singleton $\{0\}$ is open in $S$, then its preimage $\beta^{-1}(0) = (0,0)$ must be open in $(-\pi,\pi)$, which is a contradiction. We also have that $S$ is not $2$-dimensional since $(-\pi,\pi)$ is $1$-dimensional. Excluding those cases, then $S$ must be $1$-dimensional. Since the every limit points of $S$ lies in $S$, so is $0 = \limit_{t \to -\pi}\beta(t) = \limit_{t \to -\pi}\beta(t)$, then $S$ is compact.
    
    Since $S$ is connected and compact, then the Classification theorem establishes that $S$ is homeomorphic to $\bb{S}^1$, denoted by the map $F: S \to \bb{S}^1$. Let $p \in \bb{S}^1$, since $ \bb{S}^1 \backslash \{p\}$ is still a connect manifold, whereas $S\backslash \{F(p)\}$ is not, then $S$ cannot be embedded into a submanifold of $\bb{R}^2$.
\end{proof}


\begin{problem}
    Let $\gamma: \mathbb{R} \rightarrow \mathbb{T}^2$ be the curve of Example 4.20. Show that $\gamma(\mathbb{R})$ is not an embedded submanifold of the torus. [Remark: the warning in Problem 5-4 applies in this case as well.]
\end{problem}
\begin{problem}
    Suppose $M \subseteq \mathbb{R}^n$ is an embedded $m$-dimensional submanifold, and let $U M \subseteq T \mathbb{R}^n$ be the set of all unit tangent vectors to $M$ :
$$
U M=\left\{(x, v) \in T \mathbb{R}^n: x \in M, v \in T_x M,|v|=1\right\}
$$

It is called the unit tangent bundle of $M$. Prove that $U M$ is an embedded ( $2 m-1$ )-dimensional submanifold of $T \mathbb{R}^n \approx \mathbb{R}^n \times \mathbb{R}^n$. (Used on p. 147.)
\end{problem}
\begin{proof}
    
\end{proof}
\begin{problem}
    Let $F: \mathbb{R}^2 \rightarrow \mathbb{R}$ be defined by $F(x, y)=x^3+x y +y^3$. Which level sets of $F$ are embedded submanifolds of $\mathbb{R}^2$ ? For each level set, prove either that it is or that it is not an embedded submanifold.
\end{problem}
\begin{proof}
    Solving the equation $DF(x,y) = 0$ yields $(x,y)$ equals to $(0,0)$ or $\left(\frac{-1}{3},\frac{-1}{3}\right)$. We have $F(0,0) = 0$ and $F(\frac{-1}{3},\frac{-1}{3}) =\frac{1}{27}$, then for any $c \neq 0,\frac{-1}{27}$, the level set $F^{-1}(c)$ is regular, thus it is embedded submanifold of $\bb{R}^2$. We now consider two following cases:

    If $c = 0$, suppose $F^{-1}(0)$ is an embedded submanifold. Consider the equation 
    \begin{equation}
           x^3 + xy + y^3 = 0
    \end{equation}
    where $y$ is solution by $x$. Define the following polar coordinate $(x,y) \mapsto (r\cos(\phi), r\sin(\phi))$, then it follows that 
    \[
        F(x,y) = x^3 + xy + y^3 = r^2( \cos(\phi)^3 + r\sin(\phi)\cos(\phi) + \sin(\phi)^3)
    \]
    Let $I = [0,2\pi)\backslash\{\frac{3\pi}{4},\frac{7\pi}{4}\}$, and a parametrization $r: I \to \bb{R}$ for $F^{-1}(0)$ defined by
    \[
        r(\phi) = \frac{-\sin(\phi)\cos(\phi)}{\sin(\phi)^3 + \cos(\phi)^3}, \phi \in I.
    \]
    Since $(0,0)$ is the only critical point on the level set $F^{-1}(0)$, we aim to count how many times $r$ passes through this point as a curve. Let $r(\phi) = 0$, then we have 
    \[
        -\sin(\phi)\cos(\phi) = \frac{-\sin(2\phi)}{2} = 0 \lra \phi = \frac{k\pi}{2}, k \in \bb{Z}
    \]
    Since $\phi \in I$, then the following solution is $\left\{0,\dfrac{\pi}{2},\pi, \dfrac{3\pi}{2}\right\}$. 

    Now we divide the $I$ into 3 sections, which is 
    \[
       I_1 =\left(\frac{\pi}{2},\frac{3\pi}{4}\right), I_2 =\left(\pi,\frac{3\pi}{2}\right),I_3 = \left(\frac{7\pi}{4},2\pi\right)
    \]
    and define the corresponding restriction $r_i: I_i \to \bb{R}$ for $i = 1,2,3$. Then it follows that $r_1,r_2,r_3 >0$. In particular, since we have $r(I_1  + \pi) = -r(I_1)$, $r(I_2 - \pi) = -r_2(I_2)$ and $r(I_3 -\pi) = -r_3(I_3)$ . Let $J =I_1\cup I_2\cup I_3$, then the restricted map $r|_{J}: J \to r(I)$ is a homeomorphism. Moreover, since every $r_i$ is a homeomorphism onto its image, then $r_i(I_i)$ is connected. However, since $F^{-1}(0)$ is an embedded $1$-submanifold, then there exists a neighborhood $U \subseteq r(I)$ centered at $0$ such that $U$ is homeomorphic to an open interval $(-\ep,\ep)$. Since $r(I)$ can be seperated into three connected components after removing the point $0$, but removing any single point splits   $(-\ep,\ep)$ into two connected components, which leads to contradiction.

    If $c = \frac{1}{27}$, then the level set condition can be rewritten as 
    \[
        x^3 + xy + y^3 - \frac{1}{27} = \frac{1}{27}(3x + 3y - 1)(9x^2 - 9xy + 3x + 9y^2 + 3y + 1) = 0
    \]
    The solution for this equation is $x = y = \frac{-1}{3}$ and the line $L: 3x + 3y - 1 = 0$. If $F^{-1}(\frac{1}{27})$ is an embdedded submanifold, since the single seperated point$(\frac{-1}{3},\frac{-1}{3})$ is an embdedded $0$-submanifold, whether $L$ is $1$-dimension submanifold, since the general dimension cannot be different, $F^{-1}(\frac{1}{27})$ cannot be an embedded submanifold.
\end{proof}
\begin{problem}
    Suppose $M$ is a smooth $n$-manifold and $B \subseteq M$ is a regular coordinate ball. Show that $M \backslash B$ is a smooth manifold with boundary, whose boundary is diffeomorphic to $\mathbb{S}^{n-1}$. (Used on p. 225.)  
\end{problem}
\begin{proof}
    Let $\{(U_{\alpha},\varphi_{\alpha})\}$ be a smooth structure defined on $M$. Let $p$ be a point in the interior of $M \backslash B$, then there exists a smooth local chart $(U,\varphi)$ containing $p$ and $U \cap B = \varnothing$, then it is also a chart on $M \backslash B$.
    
    Let $p \in \pa{B}$. Since $B$ is a regular coordinate ball, then there exists a smooth local chart $(U_{p},\varphi_{p})$ containing $p$ such that $\varphi_p( U_p \cap B)$ is an open subset of the ball $B(0,1)$ and $\varphi(\pa B \cap U ) = \pa B(0,1) \cap \varphi(U)$. Since every neighborhood containing boundary point of $B(0,1)$ is diffeomorphic with an open subset of $\bb{H}^n$, then $U_p \cap \clo{B}$ is homeomorphic to an open subset of $\bb{H}^n$ with the boundary condition. 

     Since every constructed chart is restricted from $M$, then it also defines a smooth structure on $M\backslash B$. Thus $M \backslash B$ is a smooth manifold with boundary. By our chart construction, then it follows that $\pa (M\backslash B )= \pa B$, which is diffeomorphic to $\bb{S}^{n - 1}$.
\end{proof}
\begin{problem}
    Let $S \subseteq \mathbb{R}^2$ be the boundary of the square of side 2 centered at the origin (see Problem 3-5). Show that $S$ does not have a topology and smooth structure in which it is an immersed submanifold of $\mathbb{R}^2$.
\end{problem}
\begin{problem}
    For each $a \in \mathbb{R}$, let $M_a$ be the subset of $\mathbb{R}^2$ defined by
$$
M_a=\left\{(x, y): y^2=x(x-1)(x-a)\right\} .
$$

For which values of $a$ is $M_a$ an embedded submanifold of $\mathbb{R}^2$ ? For which values can $M_a$ be given a topology and smooth structure making it into an immersed submanifold?
\end{problem}
\begin{problem}
    Let $\Phi: \mathbb{R}^2 \rightarrow \mathbb{R}$ be defined by $\Phi(x, y)=x^2-y^2$.
    \begin{enumerate}
        \item Show that $\Phi^{-1}(0)$ is not an embedded submanifold of $\mathbb{R}^2$.
        \item Can $\Phi^{-1}(0)$ be given a topology and smooth structure making it into an immersed submanifold of $\mathbb{R}^2$?
        \item Answer the same two questions for $\Psi: \mathbb{R}^2 \rightarrow \mathbb{R}$ defined by $\Psi(x, y)= x^2-y^3$.
    \end{enumerate}
\end{problem}
\begin{problem}
    Suppose $E$ and $M$ are smooth manifolds with boundary, and $\pi: E \rightarrow M$ is a smooth covering map. Show that the restriction of $\pi$ to each connected component of $\partial E$ is a smooth covering map onto a component of $\partial M$. (Used on p. 433.)
\end{problem}
\begin{problem}
    Prove that the image of the dense curve on the torus described in Example 4.20 is a weakly embedded submanifold of $\mathrm{T}^2$.
\end{problem}
\begin{problem}
    Prove Theorem 5.32 (uniqueness of the smooth structure on an immersed submanifold once the topology is given).
\end{problem}
\begin{problem}
     Show by example that an immersed submanifold $S \subseteq M$ might have more than one topology and smooth structure with respect to which it is an immersed submanifold.
\end{problem}
\begin{problem}
     Prove Theorem 5.33 (uniqueness of the topology and smooth structure of a weakly embedded submanifold).
\end{problem}
\begin{problem}
     Prove Lemma 5.34 (the extension lemma for functions on submanifolds).
\end{problem}
\begin{problem}
     Suppose $M$ is a smooth manifold and $S \subseteq M$ is a smooth submanifold.
     \begin{enumerate}
        \item Show that $S$ is embedded if and only if every $f \in C^{\infty}(S)$ has a smooth extension to a neighborhood of $S$ in $M$. [Hint: if $S$ is not embedded, let $p \in S$ be a point that is not in the domain of any slice chart. Let $U$ be a neighborhood of $p$ in $S$ that is embedded, and consider a function $f \in C^{\infty}(S)$ that is supported in $U$ and equal to 1 at $p$.]
        \item  Show that $S$ is properly embedded if and only if every $f \in C^{\infty}(S)$ has a smooth extension to all of $M$.
     \end{enumerate}

\end{problem}
\begin{problem}
     Suppose $S \subseteq M$ is an embedded submanifold and $\gamma: J \rightarrow M$ is a smooth curve whose image happens to lie in $S$. Show that $\gamma^{\prime}(t)$ is in the subspace $T_{\gamma(t)} S$ of $T_{\gamma(t)} M$ for all $t \in J$. Give a counterexample if $S$ is not embedded.
\end{problem}
\begin{problem}
    Show by giving a counterexample that the conclusion of Proposition 5.37 may be false if $S$ is merely immersed.
\end{problem}
\begin{problem}
     Prove Proposition 5.47 (regular domains defined by smooth functions).
\end{problem}
\begin{problem}
     Prove Theorem 5.48 (existence of defining functions for regular domains).
\end{problem}
\begin{problem}
     Suppose $M$ is a smooth manifold with boundary, $N$ is a smooth manifold, and $F: M \rightarrow N$ is a smooth map. Let $S=F^{-1}(c)$, where $c \in N$ is a regular value for both $F$ and $\left.F\right|_{\partial M}$. Prove that $S$ is a smooth submanifold with boundary in $M$, with $\partial S=S \cap \partial M$.
\end{problem}